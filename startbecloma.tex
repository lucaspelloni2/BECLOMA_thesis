\toolname\footnote{the source code of \toolname\ is available at \toolurl } is still in a experimental stage. Thus, the implementation of it has been tested just with \textit{Android 4.4} and \textit{Mac OS 10.11}. 
Augmenting the compatibility of it with multiple operating systems is plan of my future agenda. 

\subsection{General Configuration}
First of all, in order to be able to use \toolname\ some external components must be installed and configured. 
The following shows the environment configurations that must be applied (they must be sequentially installed).
\begin{itemize}
\item \textbf{Brew} \\
Brew \cite{brew} is a manager for installing the missing packages for Mac OS. 
It can be installed running the following command line on a Mac OS: 
{\scriptsize
\begin{center}
\fbox{
\texttt{\$ /usr/bin/ruby -e "\$(curl -fsSL https://raw.githubusercontent.com/Homebrew/install/master/install)"
}
}
\end{center}
}%
\item \textbf{Android SDK Platform-Tool}\\
In order to easily install the Android SDK Platform-Tools component, which is required for launching \sapienz and \monkey just download and install \textit{Android Studio}, the official IDE for Android \cite{androidstudio}. 
The download link can be found at:  
\begin{center}
\url{https://developer.android.com/studio/index.html}
\end{center}
\item \textbf{pip} \\
Pip is a manager for installing python packages on a Mac OS. It is needed for installing some python dependencies used by \textsc{Sapienz}.  
It can be easily installed running the following command line: 	
\begin{center}
\fbox{
\texttt{\$brew install pip}
}
\end{center}
\begin{center}
or if it fails
\end{center}
\begin{center}
\fbox{
\texttt{\$ sudo easy\_install pip}
}
\end{center}

\end{itemize}

\subsection{Sapienz Environment Configuration}
In order to be able to perform a testing session using \sapienz, the \textit{emulators} respectively the attached \textit{devices} must have installed the \textbf{API 19} level (version code \textit{KitKat}) \cite{api19}. 
The following describes the configurations steps which have to be performed for using \textsc{Sapienz}. 
\begin{itemize}
\item\textbf{Environment variables} \\ 
Include the following $aapt$ path as environment variable in the bash profile:
\begin{center}
\fbox{
\texttt{/Users/<user\_name>/Library/Android/sdk/build-tools/19.x.x}
}
\end{center}

\item \textbf{Python dependencies} \\
Install some required python \textit{dependencies} using the \textit{requirements.txt} file located in the \sapienz\ directory: 
\begin{center}
\fbox{
\texttt{\$ sudo pip install -r requirements.txt}
}
\end{center}

\item\textbf{Python version} \\
The adequate python \textit{version}(\textit{2.7}) must be installed. 
Despite the correct version is installed, there may be a problem with the installation of the python dependencies. 
If that were the case, the following command line must be executed:  
\begin{center}
\fbox{
\texttt{\$ brew install python}
}
\end{center}
\end{itemize}
\subsection{Monkey Environment Configuration}
The only requirement to use monkey is that the package of \textit{Android SDK Platform-Tool} has been successfully installed.

\subsection{Settings}
%TODO far riferimento dove è presentato il config file a questa sezione per la descrizione dei testing parameters 
\label{usage: settings}
In order to start a testing session a set of specifications must be declared in the \Config\ file. 
This file configures the initial settings and the set of all parameters that will be processed by \toolname\ during the testing process. 
It is located in the \toolname\ source code and is renamed as \textit{config.properties}.
The most important properties have been already presented in the figure \ref{lst: config}.
In this sense, the following provides a detailed description about the testing parameters presented in the figure \ref{lst: config} \cite{monkey, sapienz}: 
\begin{itemize}
\item \textsc{Monkey}
\begin{itemize}
\item VERBOSITY ([\textit{-v, -v-v-v}]) \\
It defines the verbosity of the log. Each additional \textit{-v} increases the verbosity level. 
\item RANDOM\_EVENTS (\textit{int})\\
Number of random events that will be generated by \monkey\ in the testing cycle of a single \textit{APK}.
\item DELAY\_BETWEEN\_EVENTS (\textit{millisec})\\
It inserts a delay between the above specified number of random events. 
\item PERCENTAGE\_TOUCH\_EVENTS (\textit{int}, [0,100])\\
It adjusts the percentage of \textit{touch} events of the total number of events (\eg down-up events in the screen) 
\item PERCENTAGE\_SYSTEM\_EVENTS (\textit{int}, [0,100])\\
It adjusts the percentage of \textit{system} events of the total number of events (\eg volume controls) 
\item PERCENTAGE\_MOTION\_EVENTS (\textit{int}, [0,100])\\
It adjusts the percentage of \textit{motion} events of the total number of events (\eg random movements) 
\item IGNORE\_CRASHES (\textit{boolean}, [true,false])\\
It defines whether \monkey stops when a crash or an unhandled exception occurs. 
\end{itemize}
\item \textsc{Sapienz}
\begin{itemize}
\item SEQUENCE\_LENGTH\_MIN (\textit{int})\\
Minimum number of events that will be generated for each \sapienz cycle. 
\item SEQUENCE\_LENGTH\_MAX (\textit{int})\\
Maximum number of events that will be generated for each \sapienz cycle. 
\item SUITE\_SIZE (\textit{int})\\
The number of chromosomes (\ie test cases) that an individual (\ie test suite) owns. 
\item POPULATION\_SIZE (\textit{int})\\
Numbers of individuals in the population in the genetic algorithm. 
\item OFFSPRING\_SIZE (\textit{int})\\
The numbers of elements the offspring has in the whole test suite variation operator. 
\item GENERATION (\textit{int})\\
It sets the maximum generation number. 
\item CXPB (\textit{double}, [0,1])\\
It defines the crossover probability used by the genetic algorithm. 
\item MUTPB (\textit{double}, [0,1])\\
It defines the mutation probability used by the genetic algorithm. 
\end{itemize}
\end{itemize}

In addition to them, the following specifications must be inserted: 
\begin{itemize}
\item \textsc{EMULATOR\_NAME} \\
The name of the emulator on which the tests will be performed;
\item \textsc{AVD\_BOOT\_DELAY} (\textit{sec})\\
This time indicates how many seconds \toolname\ has to \textit{wait}, when the attached devices or the emulators get rebooted;
\item \textsc{TIME\_UNIT} (\textit{sec, min, hours})\\
It specifies the time unit recognized by the following \textit{timeouts}; 
\item \textsc{MONKEY\_TIMEOUT} \\
It indicates after how long the testing cycle of a single \textit{APK} tested with \monkey\ gets automatically interrupted so that the testing of the next app can start. These time frames can be very useful since there may happen that some devices, after generating a very large number of \textit{UI events} are not longer able to process new incoming events, freezing the entire Android system. 
\item \textsc{SAPIENZ\_TIMEOUT} \\
Same as above, only for \textsc{Sapienz}.
\item \textsc{ADB\_EXEC\_DIR}\\
It indicates the location of the executable file \textit{abd} (Android Debug Bridge), a command-line tool which facilitates the communication between the user and the Android OS. 
\end{itemize}

\subsection{\toolname\ usage}
After specifying settings and parameters in the \Config\ file, \toolname\ is ready to be started. 
The line below shows the usage of the \toolname\ command-line: 
\begin{center}
\fbox{
\small{
\texttt{\$ java -jar BECLOMA.jar [{\color{blue}-emulator} | {\color{blue}-device}] 
[{\color{blue}-monkey} | {\color{blue}-sapienz}] [timer]
}
}
}
\end{center}
The table \ref{tbl: toolarguments} lists all of the supported \toolname\ arguments and explains their meaning. 
\begin{table}[tb]
\centering
\caption{Command-line arguments supported by \toolname}
\label{tbl: toolarguments}
\begin{tabular}{|l|l|}
\hline
\rowcolor[HTML]{EFEFEF} 
\multicolumn{1}{|c|}{\cellcolor[HTML]{EFEFEF}\textbf{Supported arguments}} & \multicolumn{1}{c|}{\cellcolor[HTML]{EFEFEF}\textbf{Description}}        \\ \hline
\textit{-device}                                                           & Performs the testing on a physical attached android device               \\ \hline
\textit{-emulator}                                                         & Performs the testing on a virtual specified android emulator             \\ \hline
\textit{-monkey}                                                           & The dataset will be tested using monkey                                  \\ \hline
\textit{-sapienz}                                                          & The dataset will be tested using sapient                                 \\ \hline
\textit{-timer}                                                            & Optional.  Starts a timer for a better overview during a testing session \\ \hline
\end{tabular}
\end{table}
  


































