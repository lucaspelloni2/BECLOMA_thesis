% !TEX root =  ./lucas_thesis.tex
\section{Context}
Software testing is widely recognised as an essential part of any software development process, presenting however an extremely expensive activity; in fact, its cost has been estimated at being at least half of the entire development cost \cite{Beizer:1990:STT:79060}. \\ 
While mobile applications are becoming so widely adopted, it is still unclear if they deserve any specific testing approach for their verification and validation \cite{Amalfitano2013}.\\
Since, unlike traditional software, applications are mainly exercised by user inputs, an extremely valid approach to ensure the reliability of these applications is the GUI\footnote{Graphical User Interface} testing. In particular, in this kind of testing, each test case is designed and run in the form of sequences of GUI interaction events. \\
The most famous automated GUI testing tools and their properties are discussed in the chapter \textit{Related Work}. \\
Despite a strong evidence for automated testng approaches in verifying GUI application and revealing bugs, these state-of-art tools cannot always achieve a high code coverage \cite{Nagappan2015}. One reason is that an automated event-test-generation tool is not suited for generating inputs that require human intelligence (e.g., inputs to text boxes that expect valid passwords, or playing and winning a game with a strategy, etc.).
For this reason, sometimes a time-consuming manual approach can be needed for testing an application \cite{Nagappan2015}. \\
However, GUI testing could not be the only approach to help developers find bugs in a mobile application. Nowadays, the exponential growth of the mobile stores offers an enormous amount of informations and feedbacks from users. Therefore, another different strategy is to incorporate opinions and reviews of the end-users during the software's evolution process. \newline
In this direction, in a recent work Panichella \etal introduced a tool called SURF (Summarizer of User Reviews Feedback), that is able to analyse the useful informations contained in app reviews and to performs a systematic summarisation of thousands of user reviews through the generation of an interactive agenda of recommended software changes \cite{DBLP:conf/sigsoft/SorboPASVCG16}.

\section{Motivation}
\section{Motivation Example}
\section{Research questions}