\documentclass{seal_thesis}

% ---------------------- INSERT USED PACKAGES HERE ---------------------- 
\usepackage{xspace}
%\usepackage[usenames,dvipsnames]{xcolor}   
\usepackage[table,xcdraw]{xcolor}
\usepackage[normalem]{ulem}
\useunder{\uline}{\ul}{}
\usepackage{url}
\usepackage{amsmath,mathtools}
\usepackage{graphicx, color}                                           
\usepackage{balance}                          
\usepackage{multirow}     
\usepackage{multicol}            
\usepackage{fancyvrb}     
\usepackage{tcolorbox}
\usepackage{tabularx,ragged2e} 
\usepackage{anyfontsize}
\usepackage{amsmath}
\usepackage[TABBOTCAP]{subfigure}                         
\usepackage{tikz}                  
\usepackage{paralist}   
\usepackage{booktabs}

\usepackage{enumitem}

\setlist[enumerate,1]{%
  label=\arabic*.,
}

\newlist{inlinelist}{enumerate*}{1}
\setlist*[inlinelist,1]{%
  label=(\roman*),
}

\usepackage{amssymb}
\usepackage{ifthen}
\usepackage{array}

\makeatletter

% this  is a easy way to add and highlight new text  ...
% just comment in/out the \tnew macro ..
\newcommand*{\rom}[1]{\expandafter\@slowromancap\romannumeral #1@}
\newcommand{\tnew}[1]{{\bf { #1 }} }
%\newcommand{\tnew}[1]{{ { #1 }} }

% math and theorem definition

\newcommand{\ndef}{\stackrel{\rm def}{=}}



\pagenumbering{arabic}

% nothing i.e., no-numbering final and camera ready

%\pagestyle{empty}

% personalized alignments with specified size for tables
\newcolumntype{L}[1]{>{\raggedright\let\newline\\\arraybackslash\hspace{0pt}}m{#1}}
\newcolumntype{C}[1]{>{\centering\let\newline\\\arraybackslash\hspace{0pt}}m{#1}}
\newcolumntype{R}[1]{>{\raggedleft\let\newline\\\arraybackslash\hspace{0pt}}m{#1}}

\newboolean{showcomments}


\setboolean{showcomments}{true}

% useful to avoid size increasing for textbf command (in order to preserve alignment in tables)
\newsavebox\CBox
\def\textBF#1{\sbox\CBox{#1}\resizebox{\wd\CBox}{\ht\CBox}{\textbf{#1}}}

\ifthenelse{\boolean{showcomments}}
  {\newcommand{\nb}[2]{
    \fbox{\bfseries\sffamily\scriptsize#1}
    {\sf\small$\blacktriangleright$\textit{#2}$\blacktriangleleft$}
   }
   \newcommand{\cvsversion}{\emph{\scriptsize$-$Id: macro.tex,v 1.9 2005/12/09 22:38:33 giulio Exp $}}
  }
  {\newcommand{\nb}[2]{}
   \newcommand{\cvsversion}{}
  }


\newcommand\RQ[1]{\textbf{RQ$_#1$}}
\newcommand\LUCAS[1]{\nb{LUCAS}{#1}}
\newcommand\GIO[1]{\nb{GIO}{#1}}

\newcommand\NEW[1]{\nb{NEW}{#1}}
\newcommand\TODO[1]{\nb{TODO}{#1}}
\newcommand\UPDATE[1]{\nb{TOUPDATE}{#1}}
\newcommand\CIT[1]{\nb{Cit.}{#1}}






% === Per gli snippet =====================================
\lstset{
  backgroundcolor=\color{white},   % choose the background color; you must add \usepackage{color} or \usepackage{xcolor}
%  basicstyle=\footnotesize,        % the size of the fonts that are used for the code
  breakatwhitespace=false,         % sets if automatic breaks should only happen at whitespace
  breaklines=true,                 % sets automatic line breaking
  captionpos=n,                    % sets the caption-position to bottom
  deletekeywords={},			   % if you want to delete keywords from the given language
  escapeinside={\%*}{*)},          % if you want to add LaTeX within your code
  extendedchars=true,              % lets you use non-ASCII characters; for 8-bits encodings only, does not work with UTF-8
  frame=single,                    % adds a frame around the code
  keepspaces=true,                 % keeps spaces in text, useful for keeping indentation of code (possibly needs columns=flexible)
  language=java,                      % the language of the code
  morekeywords={},            	   % if you want to add more keywords to the set
  numbers=left,                    % where to put the line-numbers; possible values are (none, left, right)
  numbersep=5pt,                   % how far the line-numbers are from the code
  numberstyle=\tiny\color{gray}, % the style that is used for the line-numbers
  rulecolor=\color{black},         % if not set, the frame-color may be changed on line-breaks within not-black text (e.g. comments (green here))
  showspaces=false,                % show spaces everywhere adding particular underscores; it overrides 'showstringspaces'
  showstringspaces=false,          % underline spaces within strings only
  showtabs=false,                  % show tabs within strings adding particular underscores
  stepnumber=1,                    % the step between two line-numbers. If it's 1, each line will be numbered
  tabsize=2                       % sets default tabsize to 2 spaces
%  title=\lstname                   % show the filename of files included with \lstinputlisting; also try caption instead of title
}

% ---------------------- INSERT CUSTOMIZED COMMENTS HERE ---------------------- 
\newcommand\toolname[0]{BECLOMA}
\newcommand{\toolurl}{\url{https://github.com/lucaspelloni2/BA_PROJ}}
\newcommand{\dynodroid}{\textsc{Dynodroid }}
\newcommand{\monkey}{\textsc{Monkey }}
\newcommand{\sapienz}{\textsc{Sapienz }}
\newcommand{\SessionLauncher}{\textsc{SessionLauncher}}
\newcommand{\FDroidCrawler}{\textsc{FDroidCrawler}}
\newcommand{\AppTester}{\textsc{AppTester}}
\newcommand{\Config}{\textsc{ConfigurationManager}}
\newcommand{\Cmd}{\textsc{CmdExecutor}}
\newcommand{\Stream}{\textsc{StreamGobbler}}
\newcommand{\Extractor}{\textsc{CrashLogExtractor}}
\newcommand{\Crash}{\textsc{CrashLog}}
\newcommand{\Lucene}{\textsc{LuceneTokenizer}}
\newcommand{\TFIDF}{\textsc{TFIDFCalculator}}
\newcommand{\Oracle}{\textsc{Oracle}}
\newcommand{\Review}{\textsc{Review}}
\newcommand{\ReviewC}{\textsc{ReviewCollector}}
\newcommand{\Linker}{\textsc{Linker}}
\newcommand{\Facade}{\textsc{SimilarityMetricsFacade}}
\newcommand{\Writer}{\textsc{Writer}}

% ---------------------- HERE STARTS THE THESIS BODY ---------------------- 
\thesisType{Bachelor Thesis}
\date{\today}
\title{GUI usability and testing of mobile applications}
\subtitle{Example subtitle}
\author{Lucas Pelloni}
\home{18.03.1993, Zurich} % Geburtsort
\country{Switzerland}
\legi{13-722-038}
\prof{Prof. Dr. Harald C. Gall}
\assistent{Dr. Sebastiano Panichella \\ Giovanni Grano (PhD student)}
\email{lucas.pelloni@uzh.ch}
\urlT{https://github.com/lucaspelloni2/BA\_PROJ}
\begindate{08.01.2017}
\enddate{08.07.2017}

\begin{document}
\maketitle



\frontmatter


\thispagestyle{empty}

\begin{flushright}
\vspace*{3cm}
  {\Large \textit{{``Program testing can be used to show the presence of bugs, but never to show their absence.``}
  \\ \vspace{1cm} Edsger W. Dijkstra}}\\
\end{flushright}

\begin{acknowledgements}
The experience I lived with the implementation and subsequently the writing of my bachelor thesis was undoubtedly one of the most rewarding academic experiences of my life until now. During these six months, I learned to know more in-depth the programming world, appreciating more and more its challenges and its beauties.
However, the realization of my thesis would not have been possible without the help of few people. First, I would like to thank all those people who have contributed directly to the realization of my bachelor thesis. After that, those who indirectly allowed me to reach such an important goal of my life: my bachelor's degree.

First of all, I would like to express my deepest gratitude to my thesis advisor, Giovanni Grano. I would like to thank him for the patience he has shown during the past six months, for its guidance and support, for its constructive criticism and for having always encouraged me to achieve my best. I think one of the best things of my thesis was that he always allowed me to experiment and to be creative with the assigned tasks. I could  "bang my head" against the new challenges and this helped me to find the right solutions.
The final result of my thesis would not have been possible without his help. 
I would like to offer special thanks to Dr. Sebastiano Panichella for believing in me from the beginning and for giving me such a high-level thesis. Thanks for allowing me to discover a branch of computer science that I did not know so in detail before. 
I thank also my two computer science schoolmates, Ile and Erion, who helped me a lot with their knowledge about android mobile devices.
Finally, I would like to thank Professor Gall for giving me the opportunity to work together with the Software Evolution \& Architecture Lab at the University of Zurich, allowing me to use the most state-of-the-art infrastructures and technologies for my thesis. 

I would like to spend some words for those people, who have supported me during the whole bachelor degree. 
I would like to thank my parents for being there for me throughout all my bachelor track and for supporting me in times of need. 
I never could have done it without you and I hope you would be proud of me. 
%and I could explore it in a more practical way.
\end{acknowledgements}

\begin{abstract}
In last years, the massive distribution of mobile devices like smartphones, tables and more recently wearables, radically changed our social life. Since the introduction of the first modern smartphone, the iPhone in the 2007, we witnesses a gradual shift from the traditional paradigm about use of technology, entering in the called \textit{post-pc} era. 
Nowadays, the mobile market attracts always more developers and software houses. To sustain this fierce competition, they need to build high quality apps and at the same time, reach the market as soon as possible. It comes naturally to note that testing plays an important role in this process. 
Research focused for decades on traditional testing, aiming at reaching its maximum automation. However, automated testing for mobile applications presents different challenges and limitations that still need to be properly investigated.
This thesis work tries to shed some initial light into possible solutions for such problems. In particular, we focused our attention to the knowledge that can be extracted by mobile stores. Indeed, such stores represent an enormous amount of data easily available, like user reviews, and are an unmatched opportunities for software engineering research.
Our final aim is to preliminarily demonstrate how such user feedback can be in some way exploited to integrate and complement the state of art Android automated testing tools. Our results show that a noticeable set of problem can be actually detected only though user feedback. Such observation puts the first stone down for a new paradigm of \textit{user oriented testing}.
\end{abstract}

\begin{zusammenfassung}
\end{zusammenfassung}

\tableofcontents
\listoffigures
\listoftables
\lstlistoflistings

\mainmatter
\chapter{Introduction}
\label{sec:intro}
% !TEX root =  ./lucas_thesis.tex
\section{Context}
Software testing is widely recognised as an essential part of any software development process, presenting however an extremely expensive activity; in fact, its cost has been estimated at being at least half of the entire development cost \cite{Beizer:1990:STT:79060}. \\ 
While mobile applications are becoming so widely adopted, it is still unclear if they deserve any specific testing approach for their verification and validation \cite{Amalfitano2013}.\\
Since, unlike traditional software, applications are mainly exercised by user inputs, an extremely valid approach to ensure the reliability of these applications is the GUI\footnote{Graphical User Interface} testing. In particular, in this kind of testing, each test case is designed and run in the form of sequences of GUI interaction events. \\
The most famous automated GUI testing tools and their properties are discussed in the chapter \textit{Related Work}. \\
Despite a strong evidence for automated testng approaches in verifying GUI application and revealing bugs, these state-of-art tools cannot always achieve a high code coverage \cite{Nagappan2015}. One reason is that an automated event-test-generation tool is not suited for generating inputs that require human intelligence (e.g., inputs to text boxes that expect valid passwords, or playing and winning a game with a strategy, etc.).
For this reason, sometimes a time-consuming manual approach can be needed for testing an application \cite{Nagappan2015}. \\
However, GUI testing could not be the only approach to help developers find bugs in a mobile application. Nowadays, the exponential growth of the mobile stores offers an enormous amount of informations and feedbacks from users. Therefore, another different strategy is to incorporate opinions and reviews of the end-users during the software's evolution process. \newline
In this direction, in a recent work Panichella \etal introduced a tool called SURF (Summarizer of User Reviews Feedback), that is able to analyse the useful informations contained in app reviews and to performs a systematic summarisation of thousands of user reviews through the generation of an interactive agenda of recommended software changes \cite{DBLP:conf/sigsoft/SorboPASVCG16}.

\section{Motivation}
\section{Motivation Example}
\section{Research questions}

\chapter{Related Work}
\label{sec:related}
% !TEX root =  ./lucas_thesis.tex
In the following two sections, I summarize the main related works on \textit{automated testing tools for Android apps} and on \textit{the broadly usage of user reviews from app store in Software maintenance activities}. 
An overview of the recent research in the field can be found in the survey by Martin \etal \cite{Martin:tse2017}. 
\section{Automated tools for Android Testing}

Unlike traditional software, mobile applications are mainly exercised by user inputs. \\ 
In the mobile world, an extremely valid approach to ensure the realiability of these applications is the GUI\footnote{Graphical User Interface} Testing. \\ 
In particular, in this kind of testing, each test case is designed and run in the form of sequences of GUI interaction events.  \\
Depending on their exploration strategy, there are in general three approaches for creating a generation of user inputs on a mobile device \cite{dynodroid, areWeThereYet}: \textit{random testing} \cite{dynodroid, monkey}, \textit{systematic testing} \cite{evodroid} and \textit{model-based testing} \cite{mobiguitar, SwiftHand, mining}. 
\subsubsection{Fuzz testing}
When test automation does occur, it typically relies on Google's Android \textit{Monkey} command-line \cite{monkey}. Since it comes directly integrated in Android Studio, the standard IDE for Android Development, it is regarded as the current state-of-practice \cite{Mahmood2014}.\\
This tool simply generates, for the specified Android applications, pseudo-random streams of user events into the system, with the goal to stress the \textit{AUT\footnote{Application Under Test}}\cite{monkey}. \\ 
The effort required for using \textit{Monkey} is very low \cite{areWeThereYet}. Users have to specify in the command-line the type and the number of the UI events they want to generate and in addition they can establish the verbosity level of the \textit{Monkey log}. \\
The set of possible \textit{Monkey parameters} can be found in the official \textit{User Guide} for Monkey \cite{monkey}. \\
The kind of testing implemented by Monkey follows a black-box approach. 
Despite the robustness, the user friendliness \cite{areWeThereYet, dynodroid} and the capacity to find out new bugs outside the stated scenarios  \cite{monkey_2}, this tool may be inefficient if the \textit{AUT} would require some human intelligence (\textit{e.g.} a login field) for providing sensible inputs \cite{dynodroid}. \\
For this reason, \textit{Monkey} may cause highly redundant and senseless user events. Even though it would find out a new bug for a given app, the steps for reproducing it may be very difficult to follow, due also to the randomness in the testing strategy implemented by \textit{Monkey}\cite{monkey_2}. \\
\textbf{Dynodroid} \cite{dynodroid} is also a random-based testing approach. However, this tool has been discovered being more efficient than \textit{Monkey} in the exploration process  \cite{areWeThereYet}. \\
One of the reasons behind a better efficacy has been that \textit{Dynodroid} is able to generate both \textit{UI inputs} and \textit{system events} (unlike \textit{Monkey}, which can only generate UI events) \cite{areWeThereYet}. \\  
Indeed, \textit{Dynodroid} can simulate an incoming SMS message on a mobile device, a notification of another app or an request of use for available wifi networks in the neighborhood \cite{dynodroid}. All these events represent \textit{non-UI events} and they are often unpredictable and therefore difficult to simulate in a suitable context (cita?). \\
\textit{Dynodroid }views the \textit{AUT} as an event-driven program and follows a cyclical mechanism, also known as the \textit{observe-select-execute} cycle \cite{dynodroid}. First of all, it \textit{observes} which events are relevant to the \textit{AUT} in the current state, grouping they together (an event must be considered relevant if it triggers a part of code which is part of the \textit{AUT}). After that, it \textit{selects} one of the previously observed events with a randomized algorithm \cite{dynodroid, areWeThereYet} and finally \textit{executes} it. After the execution of that event it reaches a new state and can start the cycle again. \\
Another advantage of \textit{Dynodroid} compared to \textit{Monkey} is that it allows users to interact in the testing process providing UI inputs. In doing so, \textit{Dynodroid} is able to exploit the benefits of combining automated with manual testing \cite{dynodroid}.

\subsubsection{Systematic testing}
The tools using a systematic explorations strategy rely on more sophisticated techniques, such as symbolic execution and evolutionary algorithms \cite{areWeThereYet}. \\
\textbf{Sapienz} \cite{sapienz} introduced a Pareto multi-objective search-based technique to simultaneously maximize coverage and fault revelation, while minimizing the sequence lengths. \\ It combines the above mentioned random-based approach with a new systematic exploration and as mentioned in the experimental results published on \cite{sapienz}, \textit{Sapienz} is an outperformer in the automated mobile testing area. \\
Indeed, in an empirical study described on \cite{sapienz}, \textit{Sapienz} has illustrated the strength of its approach. It found from a set of 68 benchmark apps, 104 unique crashes (while \textit{Monkey} 41 and \textit{Dynodroid} 13).

\subsubsection{Model-based testing}
Model-based tools for testing Android applications are quite popular \cite{sapienz}. Most of these tools \cite{mobiguitar,guiripper, swifthand, SwiftHand, mining} generate UI events from models, which are either manually designed or created from XML configuration files \cite{sapienz}. \\
For example, \textit{SwiftHand}\footnote{https://github.com/wtchoi/SwiftHand} uses a machine learning algorithm to learn a model of the current \textit{AUT}. This final state machine model \cite{areWeThereYet} generates UI events and due their execution the app reaches new unexplored states. After that, it exploits the execution of these events to adapt and refine the model \cite{swifthand}. \textit{SwiftHand}, in a similar way to \textit{Monkey} generates only touching and scrolling UI events and is not able to generate System events \cite{areWeThereYet}.

\section{Usage of users reviews in Software maintenance activities}
The concept of app store mining was first introduced by \textit{Harman} \etal
\cite{appstoremining}. In this context, many researchers focused on the analysis of user reviews to support the maintenance and evolution of mobile applications \cite{Martin:tse2017}.


%monkey (va con random testing)

%autodiscover (va con evodroid)

\chapter{Approach}
\label{sec:approach}
\label{chapter:approach}
In this work, we aim to build an approach able to (i) automatically test Android application with modern state-of-art tools, (ii) automatically collect the arose errors (\ie stack traces), (iii) detect the unique errors (\ie the ones that derive from the same bug), (iv) link such stack traces with the user reviews that claim about the correspondent problems.           
We called this approach \textbf{BECLOMA} (\textbf{B}ug \textbf{E}xtractor, \textbf{C}lassifier and \textbf{L}inker \textbf{O}f \textbf{M}obile \textbf{A}pps).
Figure \ref{fig: becloma} depicts the main actions performed by this approach.
For giving a cleaner and more understandable explanation of how \toolname\ works, we describe in the following chapter its key features (see figure \ref{fig: becloma}). Such features represent the three main processes that \toolname\ sequentially performs.
\begin{enumerate}
\item the \textsc{Testing} uses \monkey and \sapienz to test the \textit{APKs} under test, reporting their testing results and extracting possible \textit{crashes} from the before generated test logs; 

\item the \textsc{Clustering} part investigates the similarity between the previous extracted crash logs, using different metrics and strategies, in order to collect them together and create a crash log \textit{bucket}, \ie a set of unique; 

\item the \textsc{Linking} part represents the core feature of \toolname. It preprocesses a set of given \textit{user reviews} as well as the set of the previously created crash logs, in order to preprocess them for the linking procedure. Afterwards, it investigates whether it exists a correlation between the stack traces and the user feedbacks, with the aim to link, whether possible, the reviews with the crash logs. 
\end{enumerate}
\begin{figure}[tb]
\centering 
%	\vspace{-1.5mm} 
\includegraphics[width=\columnwidth]{diagrams/becloma_approach_img} 
\caption{\toolname\ approach}
\label{fig: becloma}
\end{figure}


\section{Testing and Traces Collection}
\label{approach:testing}
The first step in the overall approach basically relies on two of the most used Android testing tools, in order to exercise the SUT under tests and collect as many failures as possible. 
First of all, \toolname\ acts as crawler in order to download the set of desired \textit{APKs} from the \textit{FDroid} API. 
Thus, it firstly reads a static structured file containing a set of android package names; then, it builds the necessary \textit{HTTP links} in order to download the correspondent \textit{APK} files. 
Once the dataset has been built and some testing parameters have been specified, the testing session can be started. 
The set of parameters that must be specified are described in detail in the section \ref{usage: settings}.
The overall approach of \toolname\ consists of three testing cycles: 
\begin{enumerate}
\item The \textbf{single app} cycle concerning the testing of a single \textit{APK}. It represents the time frame for which each \textit{APK} in the dataset is tested. After that time frame, the approach starts with the testing of the next \textit{APK} indicated in the dataset. 
\item The \textbf{dataset} cycle describing the time spent for testing the \textit{APKs} within the dataset exactly one time for that currently cycle. 
\item The \textbf{session} cycle characterizing a testing session, \ie how many dataset cycles have to be performed. 
\end{enumerate}
The figure \ref{fig: testingapproach} explains the roles of these cycles in the testing approach. 
\begin{figure}[tb]
\centering 
%	\vspace{-1.5mm} 
\includegraphics[width=10cm,height=14cm]{diagrams/testingapproach} 
\caption{Testing cycles}
\label{fig: testingapproach}
\end{figure}
The process behind the testing of a single \textit{APK} is illustrated in the figure \ref{fig: apkprocess}.
It tests systemically each \textit{APK} according to its specifications and testing parameters. The former describes the number of random events that will be sent to the app under test, how long the testing of that \textit{APK} will last, the directory where the reports will be stored, etc. 

Once the testing environment has been configured, the test can concretely start. 
After the cycle describing the testing of a single \textit{APK} has finished, \toolname\ begins with the \textit{reporting} phase. 
This step of the approach consists of saving the generated reports which contain the stack traces of the runtime exceptions with the sequence of the events that led to the failures. 
Once the reports of the test has been stored, we proceed to parse them looking for an eventually arose failure. If a crash occurred, we extract its part relative to the stack trace from the whole corpus of the log. At the end, this information is saved into a specific directory aiming at collecting all the arose failures.

\begin{figure}[tb]
\centering 
\includegraphics[width=\columnwidth]{imgs/apkprocess} 
\caption{Approach performed by \toolname\ for testing a single android app}
\label{fig: apkprocess}
\end{figure}



%that contain the stack traces of the runtime exceptions detected by the tools together with the sequence of the UI and/or system events that led to the failures [6].



\section{Clustering of The Collected Traces}
\label{approach:clustering}
As we said before at the end of the testing phase all the stack traces that have been generated are stored in a specific directory. However, it is worth to notice that with an high change, most part of such failures would refer to the same bug (in other words, the traces are duplicates).

In order to identify the \textit{unique crashes}, the second step of our approaches performs an automatic clustering of the overall bunch of logs. Ideally, at the end of such phase, each cluster should contain logs that refers to the same bug.
For this task, we rely on classic Information Retrieval (IR) techniques, collecting features about the logs and comparing them using the \textit{cosine similarity} metrics (see Section \ref{sec:cosine_similarity}).
Figure \ref{fig: clustering} shows the overall process for this task. 
\begin{figure}[tb]
\centering 
\includegraphics[width=\columnwidth]{imgs/clusteringidea} 
\caption{The idea behind the Clustering process}
\label{fig: clustering}
\end{figure}
As said before, some track traces may be overlapping \ie refers to the same bug of be even duplicates. 
Despite the trigger method, \ie the method that raised to the exception, may be the same, there may be different sequences of function calls in the stack trace, the more the analysis goes deep. 
However, they are hardly detectable and is difficult to affirm that two stack traces which have the same trigger method refer to different bugs. 
Therefore, some reports may belong to two different bug groups in the bucket. 

In order to understand better the Clustering approach, a clarification of how a crash log is structured must be done. 
For this purpose, the real structure of a crash report is illustrated in the listing~\ref{lst: ringdroid}. 
\begin{lstlisting}[caption=Structure of a crash log, basicstyle=\fontsize{6}{8}\ttfamily,label={lst: ringdroid}]
// CRASH: com.ringdroid (pid 6207)
// Short Msg: android.database.StaleDataException
// Long Msg: android.database.StaleDataException: Attempted to access a cursor after it has been closed.
// android.database.StaleDataException: Attempted to access a cursor after it has been closed.
// 	at android.database.BulkCursorToCursorAdaptor.throwIfCursorIsClosed(BulkCursorToCursorAdaptor.java:64)
// 	at android.database.BulkCursorToCursorAdaptor.getCount(BulkCursorToCursorAdaptor.java:70)
...
\end{lstlisting}
A crash log is usually structured as follows: 
\begin{itemize}
\item \textit{Line 1} represents the top of the crash log, where the concerned package name is made explicit;
\item \textit{Line 2} tells in few words the cause of the exception; 
\item \textit{Line 3} complements the cause of the exception giving a long explanation about the exception itself;
\item \textit{Line 4} represents the first line of the stack trace and contains the name and the generic cause of the exception. 
From this point, all the function calls underlying are part of the stack trace;
\item \textit{Line 5} is considered the exact reason for the exception, \ie the trigger method within the source code that caused the crash;
\item From \textit{line 6} moving gradually down until the end of the stack trace, there are other nested function calls which contain additional information about the cause of the exception. Usually, the most important ones for identifying the cause are in the first few lines. 
\end{itemize}
%\GIO{Specify that we used a search based approach to identify the correct threshold to choose whether two logs are the same or not}
In order to state whether two crash reports refer to the same bug or not, we used a search based approach to identify the correct threshold. 
Indeed, we built an \textit{Oracle} which is in charge of comparing two crash logs and is able to answer the following question: \textit{Do they refer to the same bug?}. 
For its answers it makes use of a similarity tolerance, which has been adapted on the base of our experimental results. 
Indeed, we manually created the bucket of unique crash logs and consequently adapted the threshold in the oracle in order to enable it to rightly answer its questions and so reproduce the same bucket. 

\paragraph{Preprocessing.}
The first step of the clustering approach consists of \textit{preprocessing} the crash reports in order to prepare them to be compared to each other. 
To achieve this, the approach follows a grammar-based tokenization technique implemented in the well-know \textbf{Apache Lucene} \cite{lucene} library.  
%Indeed, all words contained in the crash reports are preprocessed using \textbf{Apache}. 
In this direction, we collected all words contained in the crash reports and preprocessed them using a Lucene tokenizer, called \textit{StandardTokenizer}. 
%\GIO{Do not use too often ''the approach consists''! You can use 'it' or 'we'}
This tokenizer simply splits the word fields into lexical units using punctuation and whitespaces as split points. In addition, it removes unnecessary symbols (\eg \texttt{"//"}).
\toolname\ converts all strings to lower-case and extends the tokenizer to a further regular expression. 
This because, it is a worldwide convention that developers use CamelCase notation for writing programming words such as names of classes, names of functions or names of variables.
Since most of the words included in the crash logs are programming language keywords, \toolname\ complements such the StandardTokenizer, so that CamelCase text fields can be also split at the upper-case letters into separate words. 

\paragraph{TF-IDF.} 
Once all words inside the crash logs have been tokenized, we relied on some well know Information Retrieval techniques in order to link together the duplicated crashes.
As features for such process, we relied on the computation of the the \textit{tf-idf} \textit{(term frequency inverse document frequency)} values on the collected stack traces. 
In this direction, \toolname\ computes tf-idf numerical statistics in order to find the relevance of a word in a collection of documents (in our case, the crash logs). Tf-idf is a well-know term-weighting scheme and usually is used in information retrieval or text mining \cite{tfidf}. %Indeed, tf-idf is a way to measure the relevance, the weight of a term compared to its document or its entire document collection (in our case, the crash logs). 
The importance of a term is given by the number of times it occurs in a particular document, inversely proportional to its appearance in the entire documents collection \cite{campbell}. 
%\GIO{The tfidf is not an algorithm. That's a statistics that measure the importance of a word in a set on document. Slightly modify this section changing a bit the terminology}
Generally, a tf-idf scheme consists of three main components \cite{tfidfsimilarity}: 
\begin{itemize}
\item \textbf{TF (Term Frequency)}, \ie how many times a term appears in the currently scored document, where repeated terms indicate the topic of the document; A high TF means that the word in question has a high relevance for the document. The following simplified equation \cite{tfidf} describes a formula for calculating the term frequency:
\begin{align*}
tf_{x,y} = \frac{n_{x,y}}{\mid d_{y} \mid}
\end{align*}
where $n_{x,y}$ represents the number of occurrences of the term $t_x$ in the document $d_{y}$, while $\mid d_{y} \mid$ represents the number of words inside the document $d_{y}$.

\item \textbf{IDF (Inverse Document Frequency)}, \ie the inverse of the document frequency, that represents the number of document in which the term appears. If the same term appears in fewer documents, IDF shows a high value, if a term is very common it returns a low value. 
The equation \cite{tfidf} below describes a simplified version of the inverse document frequency formula: 
\begin{align*}
idf_{x} = \frac{\mid D \mid}{\mid \{d: t_{i} \in d\} \mid}
\end{align*}
where $\mid D \mid$ is the number of documents in the collection, while $\mid \{d: t_{i} \in d\} \mid$ represents the number of documents which contains the term $t_i$

\item \textbf{TF-IDF}, \ie the product of the two above terms. If it has a high value means that the currently scored term has a high relevance, otherwise if it returns a low value, the term has little relevance.
\begin{align*}
tfidf_{x,y} = tf_{x,y}*idf_{x}
\end{align*}

\end{itemize}
The IDF metric actually measures how important a term is. This because, a term which appears very often in a single document will have a high TF score but if this term rarely occurs in the other ones it will also have have a high IDF score and so a low TF-IDF value. This would imply that it shall not have a high relevance. Otherwise, if a word occurs very often both in a single document and in the entire collection it has a high TF score and a low IDF-value, which results in a high TF-IDF score.

At the end of this phase, \toolname\ has created for each crash log its correspondent vector space model, \ie a weighted vector term, where each term indicates a new dimension in the vector and is associated with its correspondent tf-idf value. 


\paragraph{Cosine Similarity.}
\label{sec:cosine_similarity}
To state whether two crash reports refer to the same bug, the approach computes cosine similarity between their previously created vectors space model. 
The cosine similarity is just a measure of similarity between two vectors \cite{cosine} (in our case, two normalized weighted vectors consisting of their tf-idf scores).
Usually, the resulting similarity ranges from -1 to 1, but in the case of information retrieval, since the frequency of the terms are always positive, the returned values range from 0 to 1, where 0 indicates that two documents are completely decorrelated, while 1 means that the words contained inside them are exactly the same.  
The equation describing the cosine similarity between two vectors is as follows: 
\begin{align*}
cosine\:similarity = \cos({\theta}) = \frac{A\cdot{B}}{||A||\:||B||}
\end{align*}
where, in our case $A$ and $B$ are two normalized weighted term vectors consisting of tf-idf values. 
With the term "normalized" is understood that when two weighted vectors are used to compute cosine similarity among them, for each time a word is contained within a vector but not in the other, that term is inserted into the vector that does not contain it by associating a tf-idf score of 0. Furthermore, in doing so the two vectors will have the same length by making the computation of their dot product possible. 



\section{Linking approach}
\label{approach:linking}
\label{par: infusa}
To study the correlation between user reviews and the outcomes of automated testing tools, \toolname\ assumes that the set of user feedback have been already classified in according to a a defined taxonomy and preprocessed.
In order to classify the user feedback according to a given taxonomy (and ad hoch preprocessed) we relied on an approach implemented by Palomba \etal \cite{Palomba2017}. 
Indeed, they proposed some machine learning techniques for automatically classifying a set of user reviews according to a defined taxonomy. 
%\GIO{This is not super correct. In order to classify the user feedback according to a given taxonomy (and ad hoc preprocessed) we relied on the implementation done by Fabio Palomba and Adelina Ciurumelea in the paper that is also in the references. Use such paper as a reference when citing the such approach. The INFUSE-TA tools uses this approach in order to classify the reviews to link them to the logs}.

Furthermore, it performs a systematic Information Retrieval (IR) preprocessing \cite{BaezaYates:1999} on both the user reviews and \textit{augmented} stack traces aimed at (i) correcting mistakes, (ii) expanding contractions (e.g., \textit{can’t} is replaced with \textit{can not}), (iii) filtering nouns and verbs, (iv) removing common words or programming language keywords, and (v) stemming words (e.g., \textit{aiming} is replaced with \textit{aim}). \\
The text below depicts an example of Information Retrieval preprocessing applied on a user feedback: 
\smallbreak
\emph{\small``Crashes on Messages I would give this 5 stars but it crashes every time I try to access my messages in the app. I have removed and reinstalled the app  signed in and out  even reformatted my phone. But it still crashes when I click Messages  every time without fail.''}. 
\smallbreak
Following, the review is preprocessed applying the techniques described above.  
\smallbreak
\emph{\small``crash messag i would give 5 star crash everi time i tri access messag app i remov reinstal app  sign  even reformat phone but still crash i click messag  everi time without fail''}. 
\smallbreak

In this direction, another experimental called \textbf{INFUSE-TA} (\textbf{IN}tegrator o\textbf{F} \textbf{US}er \textbf{FE}edback while Testing \textbf{A}pps) developed by a team at the Software Evolution and Architecture Lab, used the approach proposed by Palomba \etal in order to classify the reviews to link them to the crash reports. 

\paragraph{Linking between Crash Logs and Reviews}
The last step of the approach consists of linking the information from the stack traces contained in the reports to the relevant user reviews. 
However, linking the two sources of information is not at all obvious.
This because, they come from different worlds: user reviews contain natural human language which describe the overall scenario that led to a failure \cite{mernik}, while the stack traces contain technical information about the exceptions raised during the execution of a certain test case. 
To account for this aspect, the approach first removes all information that creates noise in the collected stack traces, only the name and cause of the raised exceptions are selected, \ie the first line of the stack trace (in the example \ref{lst: ringdroid} the \textit{line 4}). 
The choice of considering only some specific parts of the stack traces was driven by experimental results. 
These result illustrated how the linking accuracy was influenced by the presence/absence of this information. 
After cleaning the reports, the remaining text is \textit{augmented} with the source code methods included in the stack trace. 
This concretely means, that each method extracted from the stack trace gets compared with all methods included in the source code. 
If a correlation among them exist, the stack trace is \textit{augmented} with the body and the set of words related to the that method. 
This step extends the information from the reports with contextual information from the source code, possibly providing additional information useful for the linking process. Also in this case, the choice was not random but driven by the experimental results.
Afterwards, the approach performs the systematic Information Retrieval (IR) preprocessing \cite{BaezaYates:1999} \textbf{INFUSA-TA}.
Finally, the resulting documents are linked using the asymmetric Dice similarity coefficient \cite{BaezaYates:1999}, which is defined as
follow: 
\begin{align*}
Dice (review_j, crash_i) = \frac{|W_{review_j} \cap W_{crash_i}|}{\textit{min}(|W_{review_j}|, |W_{crash_i}|)}
\end{align*}
where $W_{review_j}$ represents the set of words composing a user review $j$, $W_{crash_i}$ is the set of words contained in an augmented stack trace $i$ and the $min$ function normalizes the Dice score with respect to the number of words contained in the shortest document between $j$ and $i$. 
The asymmetric Dice similarity returns values between [0,1]. 
In my thesis, pairs of documents having a Dice score higher than \textbf{0.5} were considered as linked by the approach.








\chapter{\toolname}
\label{chapter:tool}
This section presents \toolname, a \textit{java-based} tool aimed at helping further developers during the testing process of their Android mobile applications. \\
For giving a cleaner and more understandable explanation of how \toolname\ works, I would like to split its key features into three main categories (which should also be executed in a sequential way, in order to exploit the whole \toolname's potentiality): 
\begin{enumerate}
\item the \textsc{Testing} part is in charge of testing a given set of \textit{APKs}, reporting their testing results and extracting possible \textit{crashes} from the before generated test logs; 

\item the \textsc{Clustering} part investigates the similarity between the previous extracted crash logs, using different metrics and strategies, in order to collect they together and create a crash log \textit{bucket}; 

\item the \textsc{Linking} part represents the core feature of \toolname. It pre-processes a set of given \textit{user reviews} as well as the set of the previously created crash logs, in order to prepare and "clean" them for the linking procedure.  Afterwards, it investigates whether it exists a correlation between the stack traces and the user feedbacks, with the aim to link, whether possible, the reviews with the crash logs. 
\end{enumerate}

\section{Testing}
%----------------- FDROID CRAWLER --------------
First of all, if no set of \textit{APKs} is available yet, \toolname\ can be exploited for downloading the needed mobile applications from the \textit{F-Droid API\footnote{LINK API}}. In this direction, as shown in the picture \ref{testing}, the component \FDroidCrawler, is first in charge of  parsing a static structured file (\textit{e.g.} a \textit{csv}-file format), which contains a set of android packages names. 
The path of this file is given in the \textsc{configuration manager}, which contains a set of static properties that get elaborated by \toolname. Second, \textsc{fdroid crawler} searches and then extracts a set of \textit{HTTP links} for those android packages that have been found on the API. Afterwards, it builds the correct \textit{HTTP requests} and finally starts the downloading process, saving the returned \textit{APKs} in a given directory.

%-------------- CONFIGURATION ENVIRONMENT--------------
The first step of the testing part was to build a set of \textit{APKs}, with which to perform the testing process. As said, this can be achieved using either \textsc{fdroid crawler} or can also be manually created. Now, the second step is to prepare and configure the testing environment. 
All the parameters needed for starting a testing session have to be specified in the \textsc{configuration manager}. Figure \ref{config}, shows an example of a simplified set of parameters which must be given a priori in order to launch a testing session. \newpage
\label{config}
\begin{lstlisting}[caption=Properties which get elaborated during the testing sessions]
/**
 * Testing session specifications
 */
MINUTES_PER_APP = 30
NR_OF_ITERATIONS = 5
 
/**
* Test logs directories
*/
MONKEY_DIR = Reports/MonkeyReports
SAPIENZ_DIR = Reports/SapienzReports

/**
 * Monkey parameters
 */
LOG_VERBOSITY = -v 
PACKAGE_ALLOWED = -p
NR_INJECTED_EVENTS = 5000
DELAY_BETWEEN_EVENTS = 10
PERCENTAGE_TOUCH_EVENTS = 15
PERCENTAGE_SYSTEM_EVENTS = 15
PERCENTAGE_MOTION_EVENTS = 15
IGNORE_CRASH = True

/**
* Sapienz parameters
*/
SEQUENCE_LENGTH_MIN = 20
SEQUENCE_LENGTH_MAX = 500
SUITE_SIZE = 5
POPULATION_SIZE = 50
OFFSPRING_SIZE = 50
GENERATION = 100
CXPB = 0.7
MUTPB = 0.3
\end{lstlisting}
The figure above represents a part of the \textsc{configuration manager}, where all the testing parameters for \monkey and \sapienz are specified. An in-depth explanation about these parameters concering \monkey and \sapienz can be found on \cite{monkey}, respectively on \cite{sapienz}.\\
In addition to them, the directories on which the generated test logs are going to be stored must be given as well as the specifications about the testing session. The properties about the testing session consist of two values: 
\begin{itemize}
\item \textsc{minutes\_per\_app}, specifies how many minutes an app will be tested. After that time frame, a time-out occurs and the testing process gets restarted with the next app. 
\item \textsc{nr\_of\_iterations}, specifies how many times the whole dataset will be tested.
\end{itemize}

According to the example \ref{config} above and assuming that the \textit{APKs} set consists in 10 apps, the total estimated testing time for an entire testing session would be: 
\begin{align*}
30 \:min \: \frac{per}{app} * 10\: app * 5\: iterations = 1500 \:min. \:(25\: hours). 
\end{align*}
Once the environment testing variables have been configured, the automated tool with whom the testing is going to be performed must be made explicit. Indeed, it has to be specified as parameter in \textsc{main} \textit{args} (as mentioned before in the section \ref{sec:choicetool}, the tools which can be selected are either \monkey or \sapienz). \\
The last configuration step is to define on which kind of device (\textit{i.e}, a real device, such as a \textit{tablet} or a virtual device, such as an \textit{emulator}) the testing is going to be performed. In addition to them, an additional argument that starts a timer for a better overview during the testing process can also be passed as main argument. \toolname\ supports different types of emulators or real devices running on different android API levels. However, in order to correctly execute \sapienz, the API level shall be the \textit{Android 4.4, KitKat}. \\
The listing below shows an example of a combination of possible parameters that could be given as main arguments. 


\begin{lstlisting}[caption=\toolname\ command line, language=bash]
$ java -\toolname.jar -device -monkey -timer
\end{lstlisting}

%-----------LAUNCH A TESTING SESSION -----------
Once the configuration phase is terminated, \toolname\ is able to start the testing process. 
As shown in the picture \ref{testing}, it manages the component \SessionLauncher, which is in charge of translating the previously specified testing properties into "java readable code" and initializing the testing session. 
Concretely, after \toolname\ invokes \SessionLauncher\ all the attached devices respectively the chosen emulators get initialized, \ie they get rebooted and restarted as root, so that some important write-read-permissions are enabled during the testing session. 
Whether the timer has been given as main argument, it gets also started. \\
Once the initialization step has been completed, \SessionLauncher\  invokes the \AppTester\ component which finally starts the testing session. The Listing~\ref{lst:startsession} gives a simplified code snippet about the beginning of the testing process. 

\begin{lstlisting}[caption=\SessionLauncher\ Code snippet for starting a testing session, ,label={lst:startsession}]
private appTester; 
public void startTestingSession() throws Exception {
        final int NUMBER_ITERATIONS = ConfigurationManager.getNumberOfIterations();
        if (IS_EMULATOR) {
        	   SessionLauncher.initialiseEmulator();
        }
        else {
       	   SessionLauncher.initialiseDevices();
        }
       
        if (isTimer) {
            SessionLauncher.initializeTimer();
        }
        for (int i = 0; i < NUMBER_ITERATIONS; i++) {
            System.out.println("Iteration number " + (i+1));
            this.appTester = new AppTester();
            this.appTester.testAllApp();
        }
}
\end{lstlisting}
First of all, the total number of iterations specified in the \Config\ is read and stored into a constant of type int. After that, all the attached devices (or the chosen emulators) gets statically initialized. According to the boolean variable \textit{isTimer}, a timer may also be started.
Afterwards, a for-loop starts where at each iteration the method \textit{testAllApp()} gets invoked. 
The idea behind this, is that at each iteration a new object of type \AppTester\ is created, so that each created object represents one testing loop of the dataset. 
For this reason, as shown in the figure~\ref{testing}, the \SessionLauncher\ would be able to instantiate infinite times the class \AppTester. However, it must create at least one object of that type in order to start a testing session. 

\AppTester\ and \Cmd\ represent the core components of the whole testing process. Indeed, \AppTester\ can be viewed as brain of the process, since it tells step-by-step to the body, \ie the \Cmd\ component, which commands it has to execute and at what stage of the process it has to perform it. \\
Listing~\ref{lst:apptester} shows a very simplified code snippet of the relation between the two above mentioned components. 
\begin{lstlisting}[caption=Testing mechanism between \AppTester\ and \Cmd\, ,label={lst:apptester}]
/**
* @class: AppTester
*/
public void testAllApp() {
        for (File apk : this.apksDirectory) {
            if (apk.getName().endsWith(".apk") && !apk.isDirectory()) {
                    uninstallApp(apk.getName());
                    installApp(apk.getName());
                    if (IS_MONKEY) {
                        testAppWithMonkey(config.getMonkeyRepDir(), apk.getName());
                    } else if (IS_SAPIENZ) {
                        testAppWithSapienz(config.getSapienzRepDir(), apk.getName());
                    }
            }
        } 
        // waiting to threads to finish 
        File testLog = CmdExecutor.getCurrentLog();
        if (hasCrash(testLog)) {
            generateCrashLog(testLog);
        }
} 
private void testAppWithSapienz(String dest, final String APK_NAME) {
	CmdExecutor.generateReport(dest, CommandLines.SAPIENZ_CMD_LINE(APK_NAME)); 
}

private void testAppWithMonkey(String dest, final String APK_NAME) {
	CmdExecutor.generateReport(dest, CommandLines.MONKEY_CMD_LINE(APK_NAME)); 
}

/**
* @class: CmdExectutor
*/
public static void generateReport(String dest, String cmd){
        Runtime runtime = Runtime.getRuntime();
        Process p = runtime.exec(cmd);
        StreamGobbler output = new StreamGobbler(p.getInputStream(), cmd, dest); 
        output.start();
        writeTestingEndTime(dest);
    }
    
public static File getCurrentLog() {
    return lastGeneratedLog();
}
    
\end{lstlisting}


First of all, \AppTester\ creates a for-loop in which it iterates each \textit{APK} file contained in the \textit{APKs} directory. Once again, this directory is specified in the \Config. 
The first if statement checks whether the file in question has an adequate extension, \ie it is able to be installed on a android mobile device. After that, \AppTester\ uninstalls the concerned \textit{APK}, so that at each iteration of the testing it get reinstalled. This beacause, it may be that an \textit{APK} gets affected by previously generated sequences (\eg a sequence of random events that led the app to an external website). \\
Afterwards, \AppTester\ checks which automated tool has been chosen by the tester, so that it can tell to the \Cmd\ component, which command-line it has to execute. As stated before, \AppTester\ prepares the single testing components such as which \textit{APK}, which tool, which testing parameters, etc., while \Cmd\ executes them without any prior knowledge. \\
After the automated tool has been detected, \Cmd\ is able to concretely start the testing, executing the passed command-line. This is represented in listing \ref{lst:apptester} by the method \textit{generateReport}. Indeed, \Cmd\ uses a single instance of the java-class \textit{Runtime} that allows the application to interact with the environment in which the app is running \cite{runtime}. This is actually achieved by the \textit{Runtime.getRuntime()}. The next line executes with the previously created object the given command-line. Since this method returns a new Process object, the result of the execution is assigned to a separate process. \\
Assigning the execution of each single command-line to a new single separate process brings with it many advantages: 
\begin{itemize}
\item Processes are independent of each other. If the execution of a command-line fails, it can be interrupted without affecting the entire testing process; 
\item Multithreading can be easily supported; Indeed, the component \Stream\ extends the \textit{Thread} java-class which implements the \textit{Runnable} java-interface. Each time a new process comes in, it starts a new thread in this class.
\item Each process has it own timeout. It may be that some command-lines cannot properly terminate and need to be interrupted during their execution.  
\end{itemize}
\Stream, in turn, is in charge of writing the test report. Each time its construct get instantiated in the \textit{generateReport()} method of the \Cmd\ class, it starts a new thread and begins in parallel the writing phase of the log. 
As shown in listing~\ref{lst:gobbler}, the method \textit{run()} overridden from the \textit{Runnable} interface gets automatically per-default invoked when in the \textit{generateReport} method an object of type \textit{StreamGobbler} calls the \textit{start()} method. 
Once the \textit{start()} method is called, the writing phase starts. This phase uses a \textit{PrintWriter} as well as classic \textit{Reader} for writing text on a file.
Before the test log is written, the metadata about the testing environments are appended to the writer. At the end of the process the writer is closed and the thread can terminate. Once the thread is finished, the method \textit{writeTestingEndTime()} in the method \textit{generateReport()} can start. This method complement the metadata writing the testing end time, so that the total testing time can be computed. 





\begin{lstlisting}[caption=\Stream\ code snippet writing a test log, ,label={lst:gobbler}]
/**
* @class: StreamGobbler
*/
@Override
public void run() {
        try {
            Writer writer = new PrintWriter(outputPath, "UTF-8");
            InputStreamReader isr = new InputStreamReader(is);
            BufferedReader br = new BufferedReader(isr);
            String line;
            writer.append(TesterData.getMetaData()); // insert metadata
            while ((line = br.readLine()) != null) {
                System.out.println(" > " + line); // console overview
                writer.append(line).append("\n"); // test log 
            }
            closeWriter();
        } catch (IOException ioe) {
            ioe.printStackTrace();
        }
}
\end{lstlisting}

Listing~\ref{lst:testinglog} shows a short version of a test log of the app \textit{com.danvelazsco.fbwrapper} that has been generated after the execution of \monkey. 

\begin{lstlisting}[caption=Test log of com.danvelazco.fbwrapper, basicstyle=\fontsize{7}{8}\ttfamily,label={lst:testinglog}]
/**
 * Meta-data
 */
Tester Name: Lucas Pelloni
Testing Start Time: 05/04/2017 11:18:29
Testing End Time: 05/04/2017 11:48:30
Total Testing Time: 30 minutes (0.5 hours)
Type of testing: testing on a physical device
Device name: c0808bf731ab321
Percentage of motion events: 2.0% (number of motion events: 60 of 3000 events)
Percentage of system events: 6.0% (number of system events: 180 of 3000 events)
Percentage of touch events: 1.0% (number of touch events: 30 of 3000 events)

/**
 * Test log 
 **/
:Monkey: seed=1495075565065 count=3000
:AllowPackage: com.danvelazco.fbwrapper
:IncludeCategory: android.intent.category.LAUNCHER
:IncludeCategory: android.intent.category.MONKEY
// Event percentages:
//   0: 1.0%
//   1: 2.0%
//   2: 2.4931507%
//   3: 18.698631%
//   4: -0.0%
//   5: 31.164383%
//   6: 18.698631%
//   7: 6.0%
//   8: 2.4931507%
//   9: 1.2465754%
//   10: 16.205479%
:Switch: #Intent;action=android.intent.action.MAIN;category=android.intent.category.LAUNCHER;end
    // Allowing start of Intent { act=android.intent.action.MAIN cat=[android.intent.category.LAUNCHER]
:Sending Trackball (ACTION_MOVE): 0:(4.0,4.0)
:Sending Trackball (ACTION_MOVE): 0:(4.0,-3.0)
:Sending Trackball (ACTION_MOVE): 0:(2.0,-1.0)
:Sending Trackball (ACTION_MOVE): 0:(-5.0,2.0)
:Sending Trackball (ACTION_MOVE): 0:(-5.0,3.0)
    //[calendar_time:2017-05-05 09:48:23.894  system_uptime:717348]
    // Sending event #100
...
\end{lstlisting}

As shown in the figure above, the test logs do not only contain the test results of the tested app, but also the above mentioned meta-data for documenting and retracing the whole testing session.

The testing phase in the "strict sense", \ie the stage where the \textit{APK} gets stressed with an automated tool is over. At this point, the logs must be investigated about the possibility that some apps have generated a crash during its testing time frame. In this sense, the last part of the method \textit{testAllApp()} illustrated in the listing ~\ref{lst:apptester}, is in charge of stating whether a test log contains a crash or not. The method for checking whether a test log has collected a crash inside it is quite intuitive. This because, the syntax used by \monkey and \sapienz in their report for indicating the presence of a crash is the same. As illustrated in the listing~\ref{lst:crashlog}, a crash can be delimited using the following two \textit{Strings}: 
\begin{itemize}
\item Crash beginning: \texttt{"// CRASH: "}
\item Crash end: \texttt{"// "}
\end{itemize}

\begin{lstlisting}[caption=Crash log of com.danvelazco.fbwrapper illustrated within its test log, basicstyle=\fontsize{6}{8}\ttfamily,label={lst:crashlog}]
...
:Sending Trackball (ACTION_MOVE): 0:(3.0,3.0)
:Sending Trackball (ACTION_MOVE): 0:(-4.0,-3.0)
:Sending Trackball (ACTION_MOVE): 0:(3.0,-1.0)
// CRASH: com.danvelazco.fbwrapper (pid 4302)
// Short Msg: java.lang.NullPointerException
// Long Msg: java.lang.NullPointerException
// Build Label: samsung/espressowifixx/espressowifi:4.2.2/JDQ39/P3110XXDMH1:user/release-keys
// Build Changelist: 8291
// Build Time: 1419156873000
// java.lang.NullPointerException
// 	at com.danvelazco.fbwrapper.activity.BaseFacebookWebViewActivity.onKeyDown(BaseFacebookWebViewActivity.java:649)
// 	at com.danvelazco.fbwrapper.FbWrapper.onKeyDown(FbWrapper.java:429)
// 	at android.view.KeyEvent.dispatch(KeyEvent.java:2640)
// 	at android.app.Activity.dispatchKeyEvent(Activity.java:2433)
// 	at com.android.internal.policy.impl.PhoneWindow$DecorView.dispatchKeyEvent(PhoneWindow.java:2021)
// 	at android.view.ViewRootImpl$ViewPostImeInputStage.processKeyEvent(ViewRootImpl.java:3845)
// 	at android.view.ViewRootImpl$ViewPostImeInputStage.onProcess(ViewRootImpl.java:3819)
// 	at android.view.ViewRootImpl$InputStage.deliver(ViewRootImpl.java:3392)
// 	at android.view.ViewRootImpl$InputStage.onDeliverToNext(ViewRootImpl.java:3442)
//     ...
//
:Sending Touch (ACTION_DOWN): 0:(215.0,683.0)
:Sending Touch (ACTION_UP): 0:(163.15541,597.4464)
:Sending Touch (ACTION_DOWN): 0:(243.0,812.0)
...

\end{lstlisting} 
Indeed, the method \textit{generateCrashLog()} (~\ref{lst:generatecrash} in the \textit{testAllApp()} is in charge of extracting the crash(es) from its test log. The parsing technique used by this method is to individuate the beginning of the crash using the \texttt{"START\_CRASH"}  string. Once the start has been individuated, the loop continues to add lines of the log	until it finds the \texttt{"END\_CRASH"} string. Once the end has been reached the loop terminates and \Cmd\ writes the results into an external file, in order to extract the crash. 

\begin{lstlisting}[caption=\AppTester's method for extracting a crash log from its test log,label={lst:generatecrash}]
/**
* @class: AppTester
*/
public void generateCrashLog(File testLog) {
        ...
        ArrayList<String> crashLog = new ArrayList<>();
        Pattern start = Pattern.compile(START_CRASH);
        String line;
        while ((line = in.readLine()) != null) {
            Matcher matcher = start.matcher(line);
            if (matcher.find()) {  // crash start
                while (!line.contains(END_CRASH)) { // crash end
                    crashLog.add(line);
                    line = in.readLine();
                }
            }
        }
        CmdExecutor.writeToFile(crashLog, dest);
    }
\end{lstlisting} 


At this point, the \textit{APK} has been tested, reported, its test logs have been investigated and possible crashes have been extracted. Figure~\ref{fig: apkprocess} summarizes the four components which characterizes the testing process of one application. 
\begin{figure}[tb]
\centering 
\includegraphics[width=\columnwidth]{imgs/apkprocess} 
\caption{Four test steps performed by \toolname\ of an application}
\label{fig: apkprocess}
\end{figure}



Now, \toolname\ is able to start testing the next application (\ie the next iteration of the loop inside the method \textit{testAllApp())}. 
Once all the applications specified in the dataset have been tested exactly one time, \SessionLauncher\ can begin with the next iteration of its loop (listing ~\ref{lst:startsession}) and so start testing the whole dataset another time. 
This loop ends when the number of iterations reaches the one specified by the user. \\
In addition, during each test iteration and at the end of the whole testing session useful statistics are computed and written into external excel files, using the components \textsc{OnGoingCalculator} and \textsc{FinalCalculator}. They use the pure Java library \textit{Apache POI}, for reading and writing files in Microsoft Office formats \cite{apachepoi}. 




\section{Clustering}
%--------------CLUSTERING AIM-------------
Once the testing phase is finished, all the generated crash logs are stored in a given directory. After this phase, the only way to differentiate them inside this directory is the name of the package for which these crashes occurred. However, among these crash logs there may be a lot of redundancy, since more of them may refer to the same bug. The aim behind the Clustering phase is to create a bucket of unique crash logs. This means, that each crash log has to be compared with the others	 of the same package and according to some metrics that will be explained below, they must be smartly group together. Figure~\ref{fig: clustering} shows the idea behind the Clustering process. 
\begin{figure}[tb]
\centering 
\includegraphics[width=\columnwidth]{imgs/clusteringidea} 
\caption{The idea behind the Clustering process}
\label{fig: clustering}
\end{figure}

Some groups of crash logs may be overlapping. Despite the trigger method, \ie the method that raised to the exception, may be the same, there may be different sequences of function calls in the stack trace, the more the analysis goes deep. However, they are hardly detectable and is difficult to affirm that two stack traces which have the same trigger method refer to different bugs. 

In order to understand better the Clustering approach, a clarification of how a crash log is structured must be done. In order to do this, another example of a crash log is given in the listing~\ref{lst: ringdroid}. 
\begin{lstlisting}[caption=Structure of a crash log, basicstyle=\fontsize{6}{8}\ttfamily,label={lst: ringdroid}]
1.		// CRASH: com.ringdroid (pid 6207)
2.		// Short Msg: android.database.StaleDataException
3.		// Long Msg: android.database.StaleDataException: Attempted to access a cursor after it has been closed.
4.		// android.database.StaleDataException: Attempted to access a cursor after it has been closed.
5.		// 	at android.database.BulkCursorToCursorAdaptor.throwIfCursorIsClosed(BulkCursorToCursorAdaptor.java:64)
6. 		// 	at android.database.BulkCursorToCursorAdaptor.getCount(BulkCursorToCursorAdaptor.java:70)
7.		...
\end{lstlisting}
A crash log is usually structured as follows: 
\begin{itemize}
\item \textit{Line 1} represents the top of the crash log, where the concerned package name is made explicit;
\item \textit{Line 2} tells in few words the cause of the exception; 
\item \textit{Line 3} complements the cause of the exception giving a long explanation about the exception itself;
\item \textit{Line 4} represents the first line of the stack trace. From this point, all the function calls underlying are part of the stack trace;
\item \textit{Line 5} is considered the exact reason for the exception, \ie the trigger method that caused the crash;
\item From \textit{line 6} moving gradually down until the end of the stack trace, there are other nested function calls which contain additional information about the cause of the exception. Usually, the most important ones for identifying the cause are in the first few lines. 
\end{itemize}
Before the explanation of the Clustering process can start, a premise must be made. All the classes that will be discussed below, refer to the class diagram represented in the figure~\ref{clustering}. 


%TODO aggiungere construttore in crash log passandogli gli argomenti
First of all, \toolname\ individuates the directory in which all the previously generated crash logs have been stored. This directory is indicated again in the \Config.\\  
As shown in the listing~\ref{lst: extractor}, \toolname\ loops all the crashes contained inside it and for each of them it calls the method \textit{extractCrashLog()}, in the \Extractor\ component. 

\begin{lstlisting}[caption=\Extractor\ code snippet converting crash files into CrashLog objects,label={lst: extractor}]
/**
* @class: Main
*/
CrashLogExtractor extractor = new CrashLogExtractor();
final String CRASH_CONTAINER = ConfigurationManager.getCrashContainerDirectory();
File[] crash_container = new File(CRASH_CONTAINER).listFiles();
for (File crash : crash_container) {
	extractor.extractCrashLog(crash);
}
\end{lstlisting} 
The method \textit{extractCrashLog()} converts simple crash log files stored inside a folder into \Crash\ Java-objects. 
Indeed, for each crash log file found inside the directory, the constructor of the \Crash\ class get instantiated and so a new object of this type gets created. Each time the constructor of the \Crash\ class gets invoked, it automatically creates the structure of the \Crash-object analogously to the structure of the real crash log file. \\
For instance, the crash log described in the listing \ref{lst: ringdroid} is converted into the \Crash-object represented in the listing~\ref{lst: crashobject}.
It is printed out using the trivial Java method \textit{toString()}, which  per default gives a string representation of the object in question.  

\begin{lstlisting}[caption=\Crash-object,basicstyle=\fontsize{6}{8}\ttfamily, label={lst: crashobject}]
Crash {
	crash_path = /Users/Lucas/Desktop/UZH/BA/CrashLogCollector/crash_log_com.ringdroid.txt
	packageName = com.ringdroid
	Short = android.database.StaleDataException
	Long = android.database.StaleDataException: Attempted to access a cursor after it has been closed.
	first_java_trace_line = android.database.StaleDataException: Attempted to access a cursor after it has been closed.
	trigger_method = [android.database.BulkCursorToCursorAdaptor.getCount(BulkCursorToCursorAdaptor.java:70)]
	trigger_class = BulkCursorToCursorAdaptor
	log_lines = [// CRASH: com.ringdroid (pid 20442), // Short Msg: android.database.StaleDataException, ...]
	
}
\end{lstlisting}
%	all_stack_trace_methods = [BulkCursorToCursorAdaptor, CursorWrapper, MergeCursor,
	 						  %CursorAdapter, AdapterView, AbsListView, ViewGroup, Handler, Looper, ActivityThread, 
	 						  %Method, ZygoteInit]
%augmented_stack_traces = [android, database, stale, attempted, access, cursor, closed, bulk, adaptor, count, wrapper , merge, widget, adapter]  
As you can see in the listing above, the \Crash-object has assumed the same structure as the crash log file. Indeed, it analogously describes the \textit{package name} to which the crash belongs, the \textit{short} and the \textit{long} explanations about the exception, the \textit{trigger method}, etc.
In addition to them, the \Crash-object stores a list of strings containing the log words, where each position of the array is occupied by a different line. 
%-----------LUCENE------------
\paragraph{Preprocessing.}
The second step of the clustering process is to \textit{preprocess} the crash reports in order to prepare them to be compared to each other. 
To achieve this, all the words contained in the crash reports are preprocessed using \textbf{Apache Lucene} \cite{lucene} well-known tokenization techniques.  
Indeed, the \Crash\ class statically invokes for each line the method \textit{tokenizeLine()} in the \Lucene\ class. Listing~\ref{lst: tokenizeline} shows how it concretely works. 
\begin{lstlisting}[caption=Each line inside the crash report is tokenized using \Lucene,label={lst: tokenizeline}]
/**
* @class: CrashLog
*/
private List<String> logLines;
for (String line : this.logLines) {
	LuceneTokenizer.tokenizeLine(line);
}
\end{lstlisting} 
For the tokenization process, \toolname\ uses a well known grammar-based  Lucene-tokenizer, called \textit{StandardTokenizer}. This tokenizer simply split the word fields into lexical units using punctuation and whitespaces as split points. In addition, it removes unnecessary symbols (\eg \texttt{"//"}).
\toolname\ converts all the strings to lowercase and extends the tokenizer with a further regular expression. This because, it's a worldwide convention that developers use CamelCase notation for writing programming words such as names of classes, names of functions or names of variables.
Since most of the words included in the crash logs are programming language keywords, \toolname\ complements the StandardTokenizer of Lucene, so that CamelCase text fields can be also split at the upper-case letters into separate words. 
\clearpage




Consider the following line in the crash report (listing~\ref{lst: ringdroid}) that is passed as argument to the method \textit{tokenizeLine()}: \vspace{-0.8cm}
\begin{center}
\framebox[1\width]{{\fontsize{6.5}{6}{ \texttt{
//\hspace*{0.6cm}at android.database.BulkCursorToCursorAdaptor.throwIfCursorIsClosed(BulkCursorToCursorAdaptor.java:64)}
}}} \par
\end{center}
The following two boxes show the various tokenization steps that are performed by \toolname.
\begin{enumerate}
\item \textbf{Lucene StandardTokenizer}\\ \vspace{0.3em}
\framebox[1\width]{{\fontsize{6.5}{6}{ \texttt{at}
}}}
\framebox[1\width]{{\fontsize{6.5}{6}{ \texttt{android}
}}}
\framebox[1\width]{{\fontsize{6.5}{6}{ \texttt{database}
}}}
\framebox[1\width]{{\fontsize{6.5}{6}{ \texttt{BulkCursorToCursorAdaptor}
}}}
\framebox[1\width]{{\fontsize{6.5}{6}{ \texttt{throwIfCursorIsClosed}
}}}
\framebox[1\width]{{\fontsize{6.5}{6}{ \texttt{BulkCursorToCursorAdaptor}
}}}
\framebox[1\width]{{\fontsize{6.5}{6}{ \texttt{java}
}}}
\framebox[1\width]{{\fontsize{6.5}{6}{ \texttt{64}
}}}
\item \textbf{\toolname\ CamelCase Tokenizer} \\ \vspace{0.3em}
\framebox[1\width]{{\fontsize{6.5}{6}{ \texttt{at}
}}}
\framebox[1\width]{{\fontsize{6.5}{6}{ \texttt{android}
}}} 
\framebox[\width]{{\fontsize{6.5}{6}{ \texttt{database}
}}}
\framebox[\width]{{\fontsize{6.5}{6}{ \texttt{Bulk}
}}}
\framebox[\width]{{\fontsize{6.5}{6}{ \texttt{Cursor}
}}} 
\framebox[\width]{{\fontsize{6.5}{6}{ \texttt{To}
}}}
\framebox[\width]{{\fontsize{6.5}{6}{ \texttt{Cursor}
}}}
\framebox[\width]{{\fontsize{6.5}{6}{ \texttt{Adaptor}
}}} 
\framebox[\width]{{\fontsize{6.5}{6}{ \texttt{throw}
}}}
\framebox[\width]{{\fontsize{6.5}{6}{ \texttt{If}
}}}
\framebox[\width]{{\fontsize{6.5}{6}{ \texttt{Cursor}
}}} 
\framebox[\width]{{\fontsize{6.5}{6}{ \texttt{Is}
}}}
\framebox[\width]{{\fontsize{6.5}{6}{ \texttt{Closed}
}}}
\framebox[\width]{{\fontsize{6.5}{6}{ \texttt{Bulk}
}}} 
\framebox[\width]{{\fontsize{6.5}{6}{ \texttt{Cursor}
}}}
\framebox[\width]{{\fontsize{6.5}{6}{ \texttt{To}
}}} 
\framebox[\width]{{\fontsize{6.5}{6}{ \texttt{Cursor}
}}} 
\framebox[\width]{{\fontsize{6.5}{6}{ \texttt{Adaptor}
}}} 
\framebox[\width]{{\fontsize{6.5}{6}{ \texttt{java}
}}} 
\framebox[\width]{{\fontsize{6.5}{6}{ \texttt{64}
}}} 



\item \textbf{\toolname\ LowerCase Tokenizer} \\ \vspace{0.3em}
\framebox[1\width]{{\fontsize{6.5}{6}{ \texttt{at}
}}}
\framebox[1\width]{{\fontsize{6.5}{6}{ \texttt{android}
}}} 
\framebox[\width]{{\fontsize{6.5}{6}{ \texttt{database}
}}}
\framebox[\width]{{\fontsize{6.5}{6}{ \texttt{bulk}
}}}
\framebox[\width]{{\fontsize{6.5}{6}{ \texttt{cursor}
}}} 
\framebox[\width]{{\fontsize{6.5}{6}{ \texttt{to}
}}}
\framebox[\width]{{\fontsize{6.5}{6}{ \texttt{cursor}
}}}
\framebox[\width]{{\fontsize{6.5}{6}{ \texttt{adaptor}
}}} 
\framebox[\width]{{\fontsize{6.5}{6}{ \texttt{throw}
}}}
\framebox[\width]{{\fontsize{6.5}{6}{ \texttt{if}
}}}
\framebox[\width]{{\fontsize{6.5}{6}{ \texttt{cursor}
}}} 
\framebox[\width]{{\fontsize{6.5}{6}{ \texttt{is}
}}}
\framebox[\width]{{\fontsize{6.5}{6}{ \texttt{closed}
}}}
\framebox[\width]{{\fontsize{6.5}{6}{ \texttt{bulk}
}}} 
\framebox[\width]{{\fontsize{6.5}{6}{ \texttt{cursor}
}}}
\framebox[\width]{{\fontsize{6.5}{6}{ \texttt{to}
}}} 
\framebox[\width]{{\fontsize{6.5}{6}{ \texttt{cursor}
}}} 
\framebox[\width]{{\fontsize{6.5}{6}{ \texttt{adaptor}
}}} 
\framebox[\width]{{\fontsize{6.5}{6}{ \texttt{java}
}}} 
\framebox[\width]{{\fontsize{6.5}{6}{ \texttt{64}
}}} 
\end{enumerate}
This tokenization process is repeated for all lines of all crash reports stored in the directory. At the end of this process, each \Crash-object is complemented with a new attribute which represents the set of preprocessed words.

\paragraph{TF-IDF.} After the phase of preprocessing, the crash logs are ready to be bucketed. In order to classify them, some well-known bucketing techniques are implemented by \toolname, with the aim to deduplicate the greatest number possible of crash reports.
In this direction, \toolname\ implements a \textit{tf-idf} algorithm \textit{(term frequency–inverse document frequency)}, a well-know term-weighting scheme used in information retrieval or text mining \cite{tfidf}. Indeed, tf–idf is a way to measure the relevance, the weight of a term compared to its document or its entire document collection (in our case, the crash logs). The importance of a term is given by the number of times it occurs in a particular document, inversely proportional to its appearance in the entire documents collection \cite{campbell}.  \\
Generally, a tf-idf algorithm consists in three main components \cite{tfidfsimilarity}: 
\begin{itemize}
\item \textbf{TF (Term Frequency)}, \ie how many times a term appears in the currently scored document, where repeated terms indicate the topic of the document; A high TF means that the word in question has a high relevance for the document. The following simplified equation \cite{tfidf} describes a formula for calculating the term frequency:
\begin{align*}
tf_{x,y} = \frac{n_{x,y}}{\mid d_{y} \mid}
\end{align*}
where $n_{x,y}$ represents the number of occurrences of the term $t_x$ in the document $d_{y}$, while $\mid d_{y} \mid$ represents the number of words inside the document $d_{y}$.

\item \textbf{IDF (Inverse Document Frequency)}, \ie the inverse of the document frequency, that represents the number of document in which the term appears. If the same term appears in fewer documents, IDF shows a high value, if a term is very common it returns a low value. 
The equation \cite{tfidf} below describes a simplified version of the inverse document frequency formula: 
\begin{align*}
idf_{x} = \frac{\mid D \mid}{\mid \{d: t_{i} \in d\} \mid}
\end{align*}
where $\mid D \mid$ is the number of documents in the collection, while $\mid \{d: t_{i} \in d\} \mid$ represents the number of documents which contains the term $t_i$

\item \textbf{TF-IDF}, \ie the product of the two above terms. If it has a high value means that the currently scored term has a high relevance, otherwise if it returns a low value, the term has little relevance.
\begin{align*}
tfidf_{x,y} = tf_{x,y}*idf_{x}
\end{align*}

\end{itemize}
The IDF metric actually measures how important a term is. This because, a term which appears very often in a single document will have a high TF score but if this term rarely occurs in the other ones it will also have have a high IDF score and so a low TF-IDF value. This would imply that it shall not have a high relevance. Otherwise, if a word occurs very often both in a single document and in the entire collection it has a high TF score and a low IDF-value, which results in a high TF-IDF score. \\

\toolname\ implements a \textit{tf-idf} scheme, computing TF-IDF scores for all words within all crash logs using again the \textbf{Apache Lucene} library. \\
First of all, \toolname\ creates an index using an \texttt{IndexWriter}. The following diagram illustrates the main components of the Lucene indexing process:
\begin{figure}[tb]
\centering 
\includegraphics[width=\columnwidth]{diagrams/indexingProcess} 
\caption{Apache Lucene indexing process}
\label{fig: indexingprocess}
\end{figure}

\paragraph{Creating documents and adding fields.}
The first step \toolname\ performs is to create a set of \textit{Lucene documents}. 
To do this, it goes through all previously created \Crash-objects and convert their preprocessed words into Lucene documents. Each document must contain its group of \textit{fields} in order to be indexed.
Therefore, each time a new Lucene document is created, a set of fields must be set and enclosed inside it. 
A field can be viewed as a section of a document which can be optionally stored in the index \cite{lucenefield} and usually has three components: name, type and content. For each stored field, it is important to determine which type is best suited to the content that is going to be indexed. 
For instance, \toolname\ indexes documents which enclose \textit{text fields}, since they are already stored, tokenized and indexed \cite{lucenetextfield}. 
In order to be able to compute TF-IDF values for the indexed Lucene documents, \toolname\ must enable the property that they can have \textit{term vectors}, \ie they can store "a list of the document's terms and their number of occurrences in that document" \cite{lucenetermvector}. \\
The first steps of the Lucene indexing process can be summarized by the following code snippet, which is a simplified version of the \textit{createIndex()} method, illustrated in the diagram~\ref{clustering}.

\begin{lstlisting}[caption=\TFIDF\ describing the Lucene indexing process,label={lst: indexing}]
/**
* @class: TFIDFCalculator
*/
private void createIndex(List<CrashLog> crashLogs) throws IOException {
			// TODO: create IndexWriter
            for (CrashLog crash : crashLogs) {
                ArrayList<String> crash_log_words = crash.getSetOfWords();
                // type of field is set
                FieldType fieldType = new FieldType(TextField.TYPE_STORED); 
                // term vector is enabled
                fieldType.setStoreTermVectors(true); 
                // new document is created
                Document doc = new Document(); 
                for (String word : crash_log_words) {
                    // fields are added into the documents
                    doc.add(new Field(LConstants.FIELD_NAME, word, fieldType)); 
                }
                doc.add(new Field(LConstants.FIELD_ID, crash.getPath(), fieldType));
            }
\end{lstlisting} 
As shown in the listing above, \toolname\ goes through all the given crash logs and stores in a local list their preprocessed log words. 
Afterwards, the field type \textit{TextField} is selected. 
In the next line, the above mentioned term vector property is enabled. Then, a new Lucene document is generated. 
At this point, \toolname\ goes through each preprocessed word and at each iteration adds the current word to a new field which has just been created. The new field has a name, a content, \ie the word in question and a type, \ie the previously selected field type. Finally, the entire field is added to the document. At the end of the loop an additional field is added and used as ID for the document (in our case, the crash log path acts as unique attribute in the index). 

\paragraph{Creating IndexWriter and adding documents.}
In order to index Lucene documents, \toolname\ must generate an \textit{IndexWriter}. This because, this object acts as a core component for creating and updating indexes. 
First of all, \toolname\ creates an object of type \textit{IndexWriter}. However, the instantiating process of this class requires some supplementary configurations that must be passed to the constructor in order to create a new object of that type. 
These two information consist in (i) the directory where the index should point at and (ii) the Lucene \textit{Analyser} which is in charge of analysing the indexed documents. 
After the specification of these two information, a new object of type \textit{IndexWriter} can be created. 
Once created, all documents can be added to the index. At the end of the indexing process the writer must be closed.\\
The listing below complements the code snippet~\ref{lst: indexing} with the instantiation of the \textit{IndexWriter} class and consequentially with the addition of the documents to the index.  
\begin{lstlisting}[caption=\TFIDF\ describing the instantiation of an IndexWriter,label={lst: indexwriter}]
/**
* @class: TFIDFCalculator
*/
private void createIndex (List<CrashLog> crashLogs) throws IOException {
			// directory gets specified
		    FSDirectory dir = FSDirectory.open(new File("\Users\Lucas\Desktop\BA\Index").toPath());
		    // configuration is set
            IndexWriterConfig config = new IndexWriterConfig(new StandardAnalyzer());
            // IndexWriter gets instantiated 
            IndexWriter writer = new IndexWriter(dir, config);
	
            for (CrashLog crash : crashLogs) {
                ArrayList<String> crash_log_words = crash.getSetOfWords();
                FieldType fieldType = new FieldType(TextField.TYPE_STORED); 
                fieldType.setStoreTermVectors(true); 
                Document doc = new Document(); 
                for (String word : crash_log_words) {
                    doc.add(new Field(LConstants.FIELD_NAME, word, fieldType)); 
                }
                doc.add(new Field(LConstants.FIELD_ID, crash.getPath(), fieldType));
                // document are added to the index using an IndexWriter object
                writer.addDocument(doc);
   			}
                writer.close();
}
    
\end{lstlisting} 
As illustrated in the example above, the location where the index should point at can be inserted using the Lucene class \textit{FSDirectory}. It is a base class for Directory implementations that store index files in the file system \cite{lucenefsdir}. Next line of code, the analyser which is used for analysing the Lucene documents is defined. This can made using the Lucene class \textit{IndexWriterConfig}, which holds all the configuration that is used to create an \textit{IndexWriter}. \toolname\ defines again a \textit{StandardAnalyser}, the same used for preprocessing the crash logs.
Finally, a new \textit{IndexWriter} can be constructed per the previously configured settings. 
Once an \textit{IndexWriter} is created, all Lucene documents can be added to the index using a simple method called \textit{addDocument()}. At the end of the indexing process all preprocessed words contained inside the \Crash-objects have been converted into Lucene documents and have been indexed.


\paragraph{Computing TF-IDF scores.} At this point, the index has been created and each document has the possibility to store its vector of terms so that TF-IDF values can be computed for each word of each term of vector of each crash report. 
In this direction, \toolname\ invokes the \textit{computeScoreMap()} method in the \TFIDF\ class, which returns a hash map object. 
This hash map has as keys the unique locations of the crash logs, each one of them is associated with another hash map which contains the term vector.
Table ~\ref{tbl: scoremap} shows the structure of the hash map, taking as a reference the crash log presented in the listing ~\ref{lst: ringdroid}. 
\begin{table}[htb]
\centering
\caption{Structure of the hash map containing its term vector}
\label{tbl: scoremap}
\begin{tabular}{l|c|c|}
\hline
\multicolumn{1}{|c|}{{\color[HTML]{000000} \textit{\textbf{Crash log path (key)}}}}                                 & \multicolumn{2}{c|}{{\color[HTML]{000000} \textit{\textbf{HashMap (value)}}}} \\ \hline
\multicolumn{1}{|l|}{\textit{../Desktop/BA/CrashLogCollector/crash\_log1\_com.ringdroid.txt}} & {\textit{term (key)}}                   & { \textit{tfidf (value)}}                  \\ \hline
                                                                                                            & \hspace{0.7cm}exception\hspace{0.7cm}                             & \hspace{0.2cm}1.73\hspace{0.2cm}                                      \\ \cline{2-3} 
                                                                                                            & view                               & 5.12                                  \\ \cline{2-3} 
                                                                                                            & accessibility                         & 4.77                                  \\ \cline{2-3} 
                                                                                                            & crash                                 & 1.00                                  \\ \cline{2-3} 
                                                                                                            & scrollable                            & 1.92                                  \\ \cline{2-3} 
                                                                                                            & adapter                               & 2.44                                  \\ \cline{2-3} 
                                                                                                            & ringdroid                             & 1.00                                  \\ \cline{2-3} 
                                                                                                            & ...                             & ...                                  
\end{tabular}
\end{table}

As shown in the table above, the terms which have the highest tf-idf relevance are \textit{accessibility} and \textit{view}. Indeed, they have a high term frequency in the currently scored crash log and a low document frequency of the term in the entire collection. 
Terms such as \textit{crash} or \textit{ringdroid} have a low tf-idf weight. In fact, they are generic and irrelevant terms, since they don't give any useful information about the topic of the document. \\
\toolname\ computes tf-idf scores for all vector terms of all crash logs. Once this processed is completed, the similarity between the set of crash logs stored inside the hash map can be computed.
\paragraph{Cosine similarity.} 
In order to state whether two crash logs refer to the same bug, \ie they can belong to the same group in the bucket, \toolname\ computes cosine similarity between the previously generated vectors terms. 
The cosine similarity is just a measure of similarity between two vectors \cite{cosine} (in our case, two normalized weighted vectors consisting of their tf-idf scores).
Usually, the resulting similarity ranges from -1 to 1, but in the case of information retrieval, since the frequency of the terms are always positive, the returned values range from 0 to 1, where 0 indicates that two documents are completely decorrelated, while 1 means that the words contained inside them are exactly the same.  
The equation describing the cosine similarity between two vectors is as follows: 
\begin{align*}
cosine\:similarity = \cos({\theta}) = \frac{A\cdot{B}}{||A||\:||B||}
\end{align*}
where, in our case $A$ and $B$ are two normalized weighted term vectors consisting of tf-idf values. 
With the term "normalized" is understood that when two weighted vectors are used to compute cosine similarity among them, for each time a word is contained within a vector but not in the other, the vector that does not contain the term gets complemented with it by associating a tf-idf score of 0. In doing so furthermore, the two vectors have the same length so that their dot product can be computed.
Figure below shows how the normalization process works. 

\begin{table}[htb]
\centering
\caption{Vector terms $A$ and $B$ before normalization}
\label{tbl: beforenormal}
\begin{tabular}{ccllcc}
\cline{1-2} \cline{5-6}
\multicolumn{2}{|c|}{{\color[HTML]{000000} \textit{\textbf{Vector A}}}}     &  & \multicolumn{1}{l|}{} & \multicolumn{1}{c|}{{\color[HTML]{000000} \textit{\textbf{Vector B}}}} & \multicolumn{1}{c|}{}                \\ \cline{1-2} \cline{5-6} 
\multicolumn{1}{|c|}{\textbf{terms}} & \multicolumn{1}{c|}{\textbf{tf-idf}} &  & \multicolumn{1}{l|}{} & \multicolumn{1}{c|}{\textbf{terms}}                                    & \multicolumn{1}{c|}{\textbf{tf-idf}} \\ \cline{1-2} \cline{5-6} 
\multicolumn{1}{|c|}{handler}        & \multicolumn{1}{c|}{3.15}            &  & \multicolumn{1}{l|}{} & \multicolumn{1}{c|}{exception}                                         & \multicolumn{1}{c|}{1.73}            \\ \cline{1-2} \cline{5-6} 
\multicolumn{1}{|c|}{accessibility}  & \multicolumn{1}{c|}{1.7}             &  & \multicolumn{1}{l|}{} & \multicolumn{1}{c|}{handler}                                           & \multicolumn{1}{c|}{2.01}            \\ \cline{1-2} \cline{5-6} 
\multicolumn{1}{|c|}{crash}          & \multicolumn{1}{c|}{1.13}            &  & \multicolumn{1}{l|}{} & \multicolumn{1}{c|}{accessibility}                                     & \multicolumn{1}{c|}{4.77}            \\ \cline{1-2} \cline{5-6} 
\multicolumn{1}{|c|}{invoke}         & \multicolumn{1}{c|}{1.41}            &  & \multicolumn{1}{l|}{} & \multicolumn{1}{c|}{crash}                                             & \multicolumn{1}{c|}{1.00}            \\ \cline{1-2} \cline{5-6} 
\multicolumn{1}{|c|}{ringdroid}      & \multicolumn{1}{c|}{1.00}            &  & \multicolumn{1}{l|}{} & \multicolumn{1}{c|}{scrollable}                                        & \multicolumn{1}{c|}{1.92}            \\ \cline{1-2} \cline{5-6} 
                                     & \textbf{}                            &  & \multicolumn{1}{l|}{} & \multicolumn{1}{c|}{adapter}                                           & \multicolumn{1}{c|}{2.44}            \\ \cline{5-6} 
                                     &                                      &  & \multicolumn{1}{l|}{} & \multicolumn{1}{c|}{ringdroid}                                         & \multicolumn{1}{c|}{1.00}   \\ \cline{5-6} 
                                     &                                      &  &                       &                                                                        &                                     
\end{tabular}
\end{table}

\begin{table}[htb]
\centering
\caption{Normalized weighted vector terms $A$ and $B$}
\label{tbl: afternormal}
\begin{tabular}{|c|c|c|c|}
\hline
\multicolumn{2}{|c|}{{\color[HTML]{000000} \textit{\textbf{Vector A}}}} & \multicolumn{2}{c|}{{\color[HTML]{000000} \textit{\textbf{Vector B}}}} \\ \hline
\textbf{terms}                     & \textbf{tf-idf}                    & \textbf{terms}                    & \textbf{tf-idf}                    \\ \hline
\textbf{exception}                          & \textbf{0}                         & exception                         & 1.73                               \\ \hline
handler                            & 3.15                               & handler                           & 2.01                               \\ \hline
accessibility                      & 1.7                                & accessibility                     & 4.77                               \\ \hline
crash                              & 1.13                               & crash                             & 1.00                               \\ \hline
\textbf{scrollable}                         & \textbf{0}                         & scrollable                        & 1.92                               \\ \hline
\textbf{adapter}                            & \textbf{0}                         & adapter                           & 2.44                               \\ \hline
invoke                             & 1.41                               & \textbf{invoke}                            & \textbf{0}                         \\ \hline
ringdroid                          & 1.00                               & ringdroid                         & 1.00                               \\ \hline
\end{tabular}
\end{table}
As shown in the tables above, the vectors $A$ and $B$ after the normalization process have the same length and have been complemented with their missing words. 

\paragraph{Computing Cosine Similarity to create the bucket.}
Once the vector terms have been normalized, the cosine similarity among crash reports can be computed. \\
In this direction, \toolname\ defines a threshold in the class \Oracle\,  which represents the tolerance for evaluating the similarity between two crash reports. Indeed, if the calculated cosine similarity among them is greater than the given limit, the two crash logs are considered to describe the same bug, \ie they will belong to the same bug group in the bucket. \\
Concretely, \toolname\ invokes the method \textit{fillCrashLogBucket()}, which is in charge of comparing the crash logs, classifying them with the aim of filling the bucket. 
The bucket consists of a hash map, which has as keys a set of strings which act as a labels for the entire bug group they represent. Each label, in turn, has its own list of \Crash-objects\, which have been considered by the \Oracle\ referring to the same bug.  \\
The bucketing process iterates all crash reports and it concretely works as follows: 
\begin{enumerate}
\item The method \textit{fillCrashLogBucket()} firstly extracts the first crash log in the list and insert it into the bucket, assigning to it the label of the first bug group. 
\item Starting from the second iteration, \toolname\ computes the cosine similarity between the current crash log and all crash logs which are placed at the first position in the list of the other bug groups. 
\item Whether all computed similarities are smaller than the threshold provided by the \Oracle\, means that there is no crash logs in the bucket already which can be considered the same as the current one. For this reason, \toolname\ creates a new bug group with a new label and insert into it the current crash log. \\
Otherwise, in the event that there is at least one computed similarity which is greater than the threshold, \toolname\ extracts the group inside the bucket which shows the highest similarity with the crash log and insert it into it.
In doing so, it gets added into the group which "resemble it more". 
\end{enumerate}
The \textit{BPMN}\footnote{Business Process Model and Notation} diagram \ref{bucketing} explains and summarizes the above mentioned bucketing process provided by \toolname. \\

\begin{figure}[tb]
\centering 
%	\vspace{-1.5mm} 
\includegraphics[width=\columnwidth]{diagrams/bucketingprocess.pdf} 
\caption{BPMN diagram describing the bucketing process}
\label{bucketing}
\end{figure}

At the end of the clustering phase, all crash logs collected in the testing phase have been systematically classified inside the bucket. 

%At the end of the bucketing process, \toolname\ saves the bucket and print it out. The clustering process is now completed. 
%conclusioni fare meglio

\clearpage
\section{Linking}
As mentioned in the introduction of this chapter, the linking phase performed by \toolname\ is in charge of studying the complementarity between user feedback and the outcomes of automated testing tools, \ie the crash logs. 

\paragraph{INFUSA-TA.}
In order to link these two different sources of information, \toolname\ assumes that the set of user feedbacks have been already classified, in according to a a defined taxonomy and preprocessed.
In this direction, an experimental tool called  \textbf{INFUSE-TA} (\textbf{IN}tegrator o\textbf{F} \textbf{US}er \textbf{FE}edback while Testing \textbf{A}pps) developed by a team at the Software Evolution and Architecture Lab, provides an automatic approach for the classification of a set of user reviews according to a defined taxonomy using some machine learning techniques. 
Furthermore, it performs a systematic Information Retrieval (IR) preprocessing \cite{BaezaYates:1999} on the user reviews aimed at (i) correcting mistakes, (ii) expanding contractions (e.g., \textit{can’t} is replaced with \textit{can not}), (iii) filtering nouns and verbs, (iv) removing common words or programming language keywords, and (v) stemming words (e.g., \textit{aiming} is replaced with \textit{aim}). \\
The text below depicts an example of Information Retrieval preprocessing applied on a user feedback: 
\smallbreak
\emph{\small``Crashes on Messages I would give this 5 stars but it crashes every time I try to access my messages in the app. I have removed and reinstalled the app  signed in and out  even reformatted my phone. But it still crashes when I click Messages  every time without fail.''}. 
\smallbreak
Following, the review is preprocessed applying the techniques described above.  \smallbreak
\emph{\small``crash messag i would give 5 star crash everi time i tri access messag app i remov reinstal app  sign  even reformat phone but still crash i click messag  everi time without fail''}. 
\smallbreak

%TODO : togliere ReviewPreProcessor dal class diagram?
\paragraph{Extracting user reviews.} 
The first step in linking phase performed by \toolname\, is to read the location specified in \Config\ of the dataset which contains the set of preprocessed and classified user reviews. 
Afterwards, \toolname\ uses the \ReviewC\ component illustrated in the diagram~\ref{linking}, in order to convert all user feedbacks located in the dataset to a set of \Review\ Java-objects. 
To achieve this, \toolname\ each time it find a new review in the dataset, it instantiates the class \Review\ and creates a new object of that type.
Afterwards, this object is stored inside a \textit{ReviewMap} in the \ReviewC\ class. 
This hash map has as keys a set of names of android packages for which the reviews have been submitted. Thus, each package name is associated with its own list of \Review-objects. 
The following table depicts a shortened version of the \textit{ReviewMap}.
\begin{table}[htb]
\centering
\caption{Scheme of the Reviewmap}
\label{tbl: reviewmap}
\begin{tabular}{l|l|}
\hline
\multicolumn{1}{|c|}{\textit{\textbf{Package name (key)}}}    & \textit{\textbf{Review list (value)}} \\ \hline
\multicolumn{1}{||}{{\hspace{0.5cm}\ul \textit{com.danvelazco.fbwrapper}\hspace{0.5cm}}} & Review object 1                       \\ \hline
                                                              & Review object 2                       \\ \cline{2-2} 
                                                              & Review object 3                       \\ \cline{2-2} 
                                                              & ...                                   \\ \hline
\multicolumn{1}{|c|}{{\ul \textit{com.amaze.filemanager}}}    & Review object 1                       \\ \hline
                                                              & Review object 2                       \\ \cline{2-2} 
                                                              & ...                                   \\ \cline{2-2} 
\end{tabular}
\end{table}
%extract the crash logs stored in the bucket and systematically \textit{augment} its previously preprocessed log words with 














% cose da chiedere
% 1. pacchetto URL in conflitto con pachetto URL di seal_thesis
% 2. cosa va nell'apporach e cosa va nel tool 
% 3. correzione lingua inglese
% 4. conclusions and future work? 
%.5. deve essere consegnato stampato e rilegato ?
% 6. nome per il tool LOL
% 7. RQ3?
% 8. dove si risponde in sè alle RQ? 










\section{How to start \toolname}
First of all, a set of parameters and directories have to inserted in the static \textit{Configuration Manager} file.
% qui inserire la command-line

\begin{figure}[tb]
\centering 
%	\vspace{-1.5mm} 
\includegraphics[width=\columnwidth]{diagrams/testing.pdf} 
\caption{Class Diagram of the testing part of the tool }
\label{testing}
\vspace{-3mm} 
\end{figure}


\begin{figure}[t]
\centering 
%	\vspace{-1.5mm} 
\includegraphics[width=\columnwidth]{diagrams/clustering.pdf} 
\caption{Class Diagram of the clustering part of the tool }
\label{clustering}
\vspace{-3mm} 
\end{figure}


\begin{figure}[t]
\centering 
%	\vspace{-1.5mm} 
\includegraphics[width=\columnwidth]{diagrams/linking.pdf} 
\caption{Class Diagram of the linking part of the tool }
\label{linking}
\vspace{-3mm} 
\end{figure}

\chapter{Results and Discussion}
\label{chapter:results}
%\GIO{il contenuto per questa sezione va bene. Io cambierei giusto un poco il formato col in quale lo hai scritto, nel senso rispecchiare un po' meno la struttura del paper (se poi ce lo accettano e viene pubblicato poi la tua tesi sembra copiata dal paper :)). Ad esempio i finding puoi toglierli dai riquadri e metterli sotto forma pi� discorsiva o ristrutturarli leggermente in qualche altro modo. }
We evaluated the \toolname\ testing approach by conducting an empirical study on a smaller dataset of apps \ie 60 APKs.
Afterwards, we used the collected crash logs for evaluating the correctness and reliability of our clustering approach, forming a bucket of unique crash reports. 
Finally, we chose the 3 apps that seemed most relevant to our study with enough reports generated by the testing tools and suitable reviews for investigating the accuracy of our linking approach. 

\section{Stack Trace Extraction}
To conduct our experiment, we selected a dataset of 60 APKs, grouping them together exploiting the FDroid-crawler we built. 
We made our selection as follows: we tried to chose a dataset containing the most varied number of APKs. Indeed, we chose very popular apps which have a high number of downloads (\eg \textit{Telegram}) as well as less known apps with a low number of downloads (\eg \textit{Ringdroid}). 
The entire dataset can be found in the folder "Dataset" in the \toolname\ source code.  
Afterwards, we tested the whole dataset 3 times with the Android testing tools, i.e., \monkey and \sapienz, running each tool for 30 minutes per app.
According to the testing cycles described in Section \ref{approach:testing}, this can be translated as follows: 
\begin{enumerate}
\item Number of iteration characterizing the \textit{session cycle}: \textbf{3 iterations}; 
\item Number of apps forming the \textit{dataset cycle}: \textbf{60 apps}; 
\item Time frame describing the \textit{single app cycle}: \textbf{30 min}.
\end{enumerate}
We conducted our experiment in the following environment: 2 Samsung Galaxy Tab 8 inches, with Android Kitkat 4.4 (API 19) on which we run the automated testing tools and Mac OS 10.11 for starting \toolname.
The testing parameters we inserted in the settings file are summarized in Table \ref{tbl: chosenparameters}. 
\begin{table}[tb]
\centering
\caption{Chosen parameters for \sapienz and \monkey to conduct our empirical study}
\label{tbl: chosenparameters}
\begin{tabular}{l|l|}
\hline
\rowcolor[HTML]{EFEFEF} 
\multicolumn{1}{|c|}{\cellcolor[HTML]{EFEFEF}\textbf{Automated testing tool}} & \multicolumn{1}{c|}{\cellcolor[HTML]{EFEFEF}\textbf{Chosen parameters}} \\ \hline
\multicolumn{1}{|c|}{\cellcolor[HTML]{FFFFFF}\textit{\textbf{Monkey}}}        & \textit{verbosity = -v -v -v}                                           \\ \hline
                                                                              & \textit{random events = 3000}                                           \\ \cline{2-2} 
                                                                              & \textit{delay between events = 10}                                      \\ \cline{2-2} 
                                                                              & \textit{percentage touch events = 8}                                    \\ \cline{2-2} 
                                                                              & \textit{percentage system events = 8}                                   \\ \cline{2-2} 
                                                                              & \textit{percentage motion events = 8}                                   \\ \cline{2-2} 
                                                                              & \textit{ignore crashes = true}                                          \\ \hline
\multicolumn{1}{|c|}{\textit{\textbf{Sapienz}}}                               & \textit{min sequence = 20}                                              \\ \hline
                                                                              & \textit{max sequence = 500}                                             \\ \cline{2-2} 
                                                                              & \textit{suite size = 5}                                                 \\ \cline{2-2} 
                                                                              & \textit{population size = 50}                                           \\ \cline{2-2} 
                                                                              & \textit{offspring size = 50}                                                     \\ \cline{2-2} 
                                                                              & \textit{generation = 100}                                               \\ \cline{2-2} 
                                                                              & \textit{crossover = 0.7}                                                \\ \cline{2-2} 
                                                                              & \textit{mutation = 0.3}                                                 \\ \cline{2-2} 
\end{tabular}
\end{table}
	
The reason behind the parameters we chosen for \monkey is based on experimental results. Indeed, we conducted few demo-experiments in order to state which combination of parameters led to the greater number of crashes. 
Furthermore, we changed the percentage of the system, motion and touch events at the beginning of each \textit{dataset cycle} in order to variate our testing strategy. 
The selection behind the \sapienz parameters is consistent with the empirical study conducted by Mao \etal \cite{sapienz}.  
At the end of each {session cycle} we used our clustering approach to deduplicate the collected crash logs and form a bucket of unique ones. 


Table \ref{tbl: results} shows the results of the empirical study we conducted. 
As shown in Table, \sapienz revealed the largest number of unique crash logs in each iteration of the experiment. 
However, it should be noted that the number of total crashes revealed by \monkey in the third dataset cycle is clear greater that the ones found by \sapienz. 
This because, the percentage of \textit{system} events has conspicuously grown (from 8\% to 35\%). This caused some sequences of events of that type that could not be elaborated by the devices, leading the app to failure. 
In fact, the number of unique crashes for \monkey is not greater than the ones found by \sapienz, since they do not represent crashes caused by the application under test, but they can be categorized as native crashes. 

\begin{table}[tb]
\centering
\caption{Testing results}
\label{tbl: results}
\begin{tabular}{llcc|}
\hline
\rowcolor[HTML]{EFEFEF} 
\multicolumn{1}{|l}{\cellcolor[HTML]{EFEFEF}\textit{\textbf{Iteration number}}} & \multicolumn{1}{c}{\cellcolor[HTML]{EFEFEF}\textbf{Crashes}} & \multicolumn{1}{l}{\cellcolor[HTML]{EFEFEF}\textit{\textbf{Monkey}}} & \multicolumn{1}{l|}{\cellcolor[HTML]{EFEFEF}\textit{\textbf{Sapienz}}} \\ \hline
\multicolumn{1}{|l|}{\textit{\textbf{First dataset cycle}}}                     & \multicolumn{1}{l|}{App crashed}                             & 7                                                                    & 16                                                                     \\ \cline{1-1}
\multicolumn{1}{l|}{}                                                           & \multicolumn{1}{l|}{Unique crashes}                          & 14                                                                   & 19                                                                     \\
\multicolumn{1}{l|}{}                                                           & \multicolumn{1}{l|}{Total crashes}                           & 39                                                                   & 45                                                                     \\ \hline
\multicolumn{1}{|l|}{\textit{\textbf{Second dataset cycle}}}                    & \multicolumn{1}{l|}{App crashed}                             & 12                                                                   & 23                                                                     \\ \cline{1-1}
\multicolumn{1}{l|}{}                                                           & \multicolumn{1}{l|}{Unique crashes}                          & 21                                                                   & 28                                                                     \\
\multicolumn{1}{l|}{}                                                           & \multicolumn{1}{l|}{Total crashes}                           & 46                                                                   & 57                                                                     \\ \hline
\multicolumn{1}{|l|}{\textit{\textbf{Third dataset cycle}}}                     & \multicolumn{1}{l|}{App crashed}                             & 23                                                                   & 21                                                                     \\ \cline{1-1}
\multicolumn{1}{l|}{}                                                           & \multicolumn{1}{l|}{Unique crashes}                          & 13                                                                  & 20                                                                     \\
\multicolumn{1}{l|}{}                                                           & \multicolumn{1}{l|}{Total crashes}                           & 87                                                                   & 54                                                                     \\ \cline{2-4} 
\end{tabular}
\end{table}

\section{Linking approach}
The following Subsections present the results we obtained of our empirical study. 
Subsection \ref{a} shows the precision of our linking approach,  answering the \RQ{2}, while Subsection \ref{b} presents our evaluation about the complementarity among stack traces and users feedback, answering \RQ{3}. 
\clearpage
\subsection{Precision Linking Approach}
\label{a}
\RQ{2} \textit{To what extent can we link the defects arose from automated testing tools?}
\smallbreak
To answer the \RQ{2}, we exploited the linking procedure provided by \toolname. 
Concretely, we selected from the previously tested dataset the 3 apps which seemed most relevant to our study with enough unique crash reports generated by \monkey and \sapienz and a set of adequate reviews. 
The apps we chosen are \textit{com.amaze.filemanager}, \textit{com.danvelazsco.fbwrapper} and \textit{com.ringdroid}.
We limited our study to a small group of apps because the time constraints and the high effort to manually validate the data. 
\begin{table}[tb]
	\centering
%	\caption{Precision of the Linking and Complementarity of Reviews and Stack Traces}
	\caption{Precision of the linking procedure}
%		\vspace{-2mm}
	\label{tbl: precision}
	\begin{tabular}{ccccc}
		\toprule
		\textbf{App} & \textbf{Precision}  \\ 
		\midrule
		\textit{com.amaze.filemanager} & 60\% \\
		\textit{com.danvelazsco.fbwrapper} & 72\% \\
		\textit{com.ringdroid} & 64\% & \\
		\hline
		\textbf{Average} & \textbf{65\%} \\
		\bottomrule
	\end{tabular}
	\vspace{-2.5mm}
\end{table}
Table \ref{tbl: precision} shows the precision values obtained by the linking approach implemented by \toolname\ over our limited dataset. 
The results reported in this Table confirm the accuracy and reliability of our linking procedure. 
Indeed, it shows its effectiveness in achieving quite high precision scores, \ie the percentage of correctly retrieved links was \textbf{65\%}.
We can affirm, that our approach performs in an accurate manner, providing a precise correlation between stack traces and crash-related user feedbacks. 

\subsection{Complementarity of Stack Traces and User Reviews}
\label{b}
\RQ{3} \textit{How complementary are the two source of information? Can we leverage on both of them to increase the effectiveness of the testing process?}
\smallbreak
To answer \RQ{3}, we firstly identified those links which correctly correlated a stack trace with its correspondent user review.  
Secondly, we categorized different type of issues concerning the crash-related user reviews and the stack traces, computing the followings metrics: 
\begin{itemize}
	\item $I_C$: \% of issues reported in both reviews and crash logs;
	
	\item $I_R$: \% of issues reported only in user reviews; 
	 %\ie the number of ;
	
	\item $I_T$: \% of issues reported only in crash logs. 
\end{itemize}
In order to compute these metrics, we first defined $T_{issues}$ as the total number of issues
discovered for a given app.
Therefore, consider $L_{pos}$ as the number of unique true positive links between crash reports and crash-related user feedbacks revealed by \toolname. 
Afterwards, let be $C_{logs}$ the number of crash reports which have been extracted but remain unlinked to any review.
Then, consider $R_{crash}$ as the number of user reviews, for which there exists no correlation to any crash report. 
Finally, we can define the relation between the three above proposed metrics as follows: 
$$T_{issues} = L_{pos} + C_{logs} + R_{crash} $$
%
Thus, we can formally introduce the three overlap metrics introduced above as follow: 
	\begin{equation*}
	I_C = {L_{pos}\over T_{issues}}
	\qquad
	I_R = {R_{crash}\over T_{issues}}
	\qquad
	I_T = {C_{logs}\over T_{issues}}
	\end{equation*}
	


\begin{table}[tb]
	\centering
%	\caption{Precision of the Linking and Complementarity of Reviews and Stack Traces}
	\caption{Complementarity of Reviews and Stack Traces}
%		\vspace{-2mm}
	\label{tbl: metrics}
	\begin{tabular}{ccccc}
		\toprule
		\textbf{App} & $\mathbf{I_C}$ & $\mathbf{I_R}$ & $\mathbf{I_T}$ \\ 
		\midrule
		\textit{com.amaze.filemanager} & 17\% & 50\% & 33\% \\
		\textit{com.danvelazsco.fbwrapper} & 45\% & 45\% & 10\% \\
		\textit{com.ringdroid} & 24\%	 & 62\% & 14\%\\
		\hline
		\textbf{Average} & \textbf{29\%}& \textbf{52\%}& \textbf{19\%}\\
		\bottomrule
	\end{tabular}
	\vspace{-2.5mm}
\end{table}


Table \ref{tbl: metrics} shows the achieved results for our reduced dataset. 
As we can see from the it, the results concerning the $I_R$ values (\ie the percentage of issues we can only find in user reviews) show in average greater scores than the $I_T$ values (\ie the percentage of issue encountered uniquely through automated tools).
Indeed, the number of issues reported only in user reviews is, on average, 52\%. This percentage is conspicuously greater than the number expressing the other two metrics (in fact, $I_C$ shows 29\% and $I_C$ only 19\%). \\
From these results, we can draw an unambiguous conclusion, \ie that 
the number of issues highlighted by the stack traces ($I_T$), obtained through the execution of automated testing tools, is substantially \textit{smaller} that the one identifying the number of issues reported in user reviews ($I_C$). 
Thus, we can summarize our first finding as follows: 
\begin{center}
\vspace*{-1em}
$I_T$ < $I_C$ < $I_R$ 
\vspace*{-1em}
\end{center} 
As highlighted in Table \ref{tbl: metrics}, the number of common issues detected when relying on both user reviews and stack traces is, on average, only 29\%. 
To find the cause of a such low percentage, we manually analysed the user reviews of our reduced dataset. 
Indeed, we found that the content of the reviews tend to describe the scenario of a crash which occurs during a particular input event.
However, such scenario may imply some sensible sequences of events, which are hard to randomly replicate by automated testing tools. \\
For instance, for the \texttt{com.danvelazsco.fbwrapper} app, the review shown below claims about a crash occurring during a particular swipe input event, which is probably hardly replicable by automated testing tools. 
\smallbreak
\emph{\small``Quick fix for messages crash Slide in from the right go to preferences and use either desktop version or basic version...''}
\smallbreak
Similarly, for the \texttt{com.ringdroid} app an user claim that:
\smallbreak
\emph{\small``Force closes when I search Maybe the problem is my large library but it truly is unusable...''}
\smallbreak
In this case, the user imputes the crash to his large library: obviously such particular conditions are difficult to be reproduced with an automated testing tools.
From these considerations, we can draw our second conclusion. 
In this sense, we can conclude that issues concerning user reviews are conceptually different to issues detected by the stack traces. 
However, despite the number of issues revealed by user feedback is the greatest, there can exist a sort of complementary between issues detected by stack traces and users feedback. 
Indeed, as highlighted in Table \ref{tbl: metrics}, \texttt{com.danvelazsco.fbwrapper} shows an example of that complementarity. 

With our findings we believe that we confirmed the importance and relevance to integrate user reviews in the validation process of a mobile application. 
Our results are quite promising, since they concretely showed the complementary between both source of information. 
However, there are still some limitations in the automated testing tools, since in many cases they are not able to reproduce some sensible scenarios describe in the user feedback. 
We argue that, we might augment the information in stack traces with the aim to link more crash-related user reviews with their correspondent crash reports. 




\chapter{Conclusions and Future Work}
\label{chapter:conclusion}
In my thesis we investigated the importance of integrating user reviews in the validation process of mobile applications. 
For this purpose, we introduced an approach called \toolname\ which is able to (i) test a set of apps and extract possible crashes, (ii) form a bucket of unique crash reports and 
(iii) investigate the complementarity between crash-related user reviews and the bucketed stack traces. 
With the aim to evaluate the performance of \toolname, we performed an empirical study 
over a dataset containing different types of mobile applications. 
First, we downloaded our sample of \textit{APKs} from the \textit{FDroid API} and we launched an experiment on them. 
Afterwards, we reported its results, \ie the total number of crashes and unique crashes which occurred during the testing phase. 
Then, we created a bucket of unique crash logs. For this purpose, we first manually built an oracle according to the crash reports we collected. 
Afterwards, we adapted its threshold in order to reproduce that bucket. 
Finally, we started our linking procedure by linking the crash-related user reviews and the stack traces we bucketed. 
Again, we reported its results in the chapter \ref{chapter:results}. 
 
 
We strongly believe, that our findings might convince other mobile developers to perform such an \textit{user-oriented testing} by validating the reliability of their mobile applications. 
Indeed, despite \toolname\ still remains in an experimental stage, we argue that our results are quite promising. 
However, we are conscious that we tested our approach with a small dataset and thus we may adapt both the clustering and the linking threshold whether \toolname\ would be used with another, maybe greater set of apps. 
With a new set of apps, we believe that \toolname\ would require some manual effort for setting the correct threshold. 

We are convinced that our findings and preliminary results lay the foundations for further research  into the field of \textit{integration of user feedback into the testing process}. 
At this stage, \toolname\ operates with a relative small number of user reviews and mobile applications. It would be nice to test the effectiveness of its approach with a very large amount of both sources of information. \\

There are different directions to investigate for future work. 
First, it could be possible to introduce a new tool which would be able to (i) \textit{summarize} stack
traces and user reviews linked together, supporting the activities performed by developers in their bug fixing sessions. 
(ii) create a \textit{prioritisation scheme} for the generated failures taking into account the user reviews and finally (iii) \textit{generate} specific \textit{test cases} directly from user reviews. \\
Furthermore, we plan to improve the \textit{augmenting process} of the stack traces implemented by \toolname. 
Indeed, we are convinced that a better selection of the words which augment the stack traces would improve the linking scores between crash-related user and these stack traces. 
For instance, we could consider not only the source code methods included in the stack traces but also the common classes among them. 
This would imply a greater set of words for the stack traces which may results in a better linking score. 


Finally, we could implement a further feature for \toolname\, which is in charge of additionally  filtering the reviews. 
Indeed, our approach considers also these reviews which present a positive connotation, \ie those that not refer to any crash but just express a positive opinion about the given app. 
For instance, we could try to conceive a filter function which analyses the reviews and discards, according to an external text file containing some stop words, those which express a position opinion about the app. 






 




\backmatter
%alpha
\bibliographystyle{abbrv}
\bibliography{biblio}

\end{document}


%RQ2 -> risposta precisione table II paper 
%RQ3 -> IC, IR, IT -> fare leva sulle review per aumentare l'efficacia del testing 





