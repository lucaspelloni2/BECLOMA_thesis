% !TEX root =  ./lucas_thesis.tex
Unlike traditional software, mobile applications are mainly exercised by user inputs. \\ 
In the mobile world, an exremely valid approach to ensure the realiability of these applications is the GUI\footnote{Graphical User Interface} Testing. \\ 
In particular, in this kind of testing, each test case is designed and run in the form of sequences of GUI interaction events.  \\
Depending on their exploration strategy, there are in general three approaches for creating a generation of user inputs on a mobile device \cite{dynodroid, areWeThereYet}: \textit{random testing} \cite{dynodroid, monkey}, \textit{systematic testing} \cite{evodroid} and \textit{model-based testing} \cite{mobiguitar, guidedgui, mining}. 
\subsubsection{Fuzz testing}
The most widely used tool in practice for testing Android apps with a random approach is \textit{Monkey} \cite{monkey}, the official Android testing command-line tool directly provided by Google. \\
This tool simply generates, for the specified attached devices, pseudo-random streams of user events into the system, with the goal to stress the \textit{AUT\footnote{Application Under Test}}\cite{monkey}. \\ 
The kind of testing implemented by Monkey follows a black-box approach. 
Despite the robustness, the user friendliness \cite{dynodroid} and the capacity of find out new bugs outside the stated scenarios  \cite{monkey_2}, this tool may be inefficient if the \textit{AUT} would require some human intelligence (\textit{e.g.} a login field) for providing sensible inputs \cite{dynodroid}. \\
For instance, \textit{Monkey} may cause highly redundant and senseless user events, which may be very  time consuming. \\
Even though it would find out a new bug for a given app, the steps to reproduce the bug may be very difficult to recreate, due also of the randomness of the executed tests. 





%monkey (va con random testing)

%autodiscover (va con evodroid)